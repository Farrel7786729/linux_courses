% common part for every lection

\documentclass{beamer}

\usetheme{Warsaw}
\usefonttheme[onlylarge]{structurebold}
\setbeamerfont*{frametitle}{size=\normalsize,series=\bfseries}
\setbeamertemplate{navigation symbols}{}

\usepackage{pdfpages}
\usepackage{soul}
\usepackage{ucs}

% encoding settings
\usepackage[T2A]{fontenc}
\usepackage{xltxtra}
\usepackage{hyperref}
\usepackage{polyglossia}
\usepackage{xecyr}

\setdefaultlanguage{russian}

% fonts settings
\usepackage{fontspec}
\setmainfont[Ligatures=TeX]{CMU Serif}                 % Computer Modern Unicode font
\setsansfont[Ligatures=TeX]{CMU Sans Serif}
\setmonofont{CMU Typewriter Text}

\usepackage{listings}
\lstset{frame=single,breaklines=true,language=sh}

% for strike-out text
\usepackage[normalem]{ulem}

% tabulation aka \tbs
\newcommand{\tbs}{\tt\char`\\}

% \regex for regular expressions
\newcommand{\regex}[1]{ %
\expandafter{$\ulcorner{\color{blue}\texttt{#1}}\lrcorner$} %
}

% \qatrainingtitle
\newcommand{\qatrainingtitle}[2] {%
\title[SaM Solutions. Linux QA Training] %
{ %
  Часть #1.\\%
  #2 %
}

}

\newcommand{\firstframe}{ %
\begin{frame} %
  \titlepage %
\end{frame} %
}

\author[Author, Vlad Shakhov]{Влад 'mend0za' Шахов\\Linux \& Embedded Team Leader}

\institute[SaM Solutions]
{
  Linux \& Embedded Department
}

\date[Dec 2012]

\subject{Linux QA training}

\pgfdeclareimage[height=1.5cm]{sam-solutions-logo}{clipart/sam-solutions-elinux}

\logo{\pgfuseimage{sam-solutions-logo}}

\graphicspath{{./clipart/}}



\qatrainingtitle{5}{Файловая система и права доступа.\\Пользователи, группы.}

\lstset{basicstyle=\tiny}

\begin{document}

\firstframe

\section{Владельцы файлов}

\begin{frame}[fragile]{Владельцы файлов.Определения}

  В UNIX (и Linux) любой файл имеет двух владельцев:
  \begin{enumerate}
    \item владельца-пользователя
    \item владельца-группу.
  \end{enumerate} 
  \lstinputlisting{samples/ls-l} 
  \pause

  \begin{itemize}
    \item \alert{Группой} - определенный список пользователей системы.
    \item Пользователь может быть членом нескольких групп. 
    \item Одна группа пользователя является первичной\footnote{primary (eng.)}, а остальные - дополнительными.
  \end{itemize}
  \small{Замечание: владелец-пользователь не обязан входить в группу-владельца.}
  \lstinputlisting{samples/id}

\end{frame}

\begin{frame}[fragile]{Владельцы файла. Внутреннее представление}
  \begin{itemize}
    \item \alert{Владелец-пользователь} и \alert{владелец-группа} файла определяются по идентификаторам \alert{UID}\footnote{UID - User IDentifier} и \alert{GID}\footnote{GID - Group IDentifier}, а не по именам. 
\lstinputlisting{samples/ls-nl} \pause
    \item Зарегистрированные пользователи и группы определены в файлах \alert{/etc/passwd} и \alert{/etc/group} \newline
    Следствие: могут существовать файлы, принадлежащие не зарегистрированным в системе пользователям и группам. \pause
    \item Новые файлы создаются с владельцем-пользователем, запустившим команду создания и его первичной группой, как владельцем группой. \pause
  \end{itemize}

\end{frame}

\begin{frame}[fragile]{Управление владельцами}

  Команды изменения владельцев
  \begin{itemize} 
    \item \alert{chown} : смена владельца-пользователя \newline
      Пример: \verb+chown sys something.doc+
    \item \alert{chgrp} : смена владельца-группы \newline
      Пример: \verb+chgrp adm file.txt+
  \end{itemize} \pause

  Кто может сменить владельцев?
  \begin{itemize}
    \item владельца-пользователя  - текущий владелец
    \item владельца-группу может владелец-пользователь для группы, к которой он сам принадлежит
    \item  администратор (root, UID=0)\footnote{На пользователя c UID=0 не распространяются ограничения прав доступа} 
  \end{itemize} 
\end{frame}

\begin{frame}{Практика: Управление владельцами файла}
  
  \alert{Упражнение 1.} Узнать имена своих групп, первичную группу.\newline
  \alert{Упражнение 2.} В папке /tmp создать файл с произвольным именем. 
  Сменить группу-владельца файла на другую (из списка своих групп). 
  Проверить доступность файла на редактирование и удаление.\newline
  \alert{Упражнение 3.} В папке /tmp создать файл с произвольным именем.
  Сменить пользователя-владельца файла на другого (одного из определённых в системе).  Проверить доступность файла на редактирование и удаление. \newline
  \alert{Упражнение 4.} В папке /tmp создать каталог с произвольным именем. Скопировать в него несколько файлов из своего домашнего каталога. Рекурсивно сменить группу-владельца (см Упражнение 2) и пользователя-владельца (см Упражнение 3). Проверить доступ к созданным файлам и каталогам.

\end{frame}

\section{Права доступа}

\subsection{Аттрибуты}

\begin{frame}{Введение в права доступа}
  У каждого файла присвоены атрибуты, называемые \alert{правами доступа}.\newline
  
  Проверяются при каждом обращении к любому файлу с любой операцией (чтение, запись, выполнение).\newline \pause

\end{frame}


\begin{frame}{Классы доступа и права доступа}
  В UNIX три базовых типа (класса) доступа:
  \begin{enumerate}
    \item \alert{u} (user) для владельца-пользователя
    \item \alert{g} (group) для владельца-группы         
    \item \alert{o} (other) для всех остальных \pause
    \item \alert{а} (all) - объединяет 3 предыдущих класса. Для всех классов пользователей
  \end{enumerate} \pause 
  .\newline 
  Три основных права доступа для каждого из классов:
  \begin{enumerate}
    \item \alert{r} (read) право на чтение           
    \item \alert{w} (write) право на запись           
    \item \alert{x} (execute) право на выполнение  
  \end{enumerate}

\end{frame}

\begin{frame}[fragile]{Разбор прав доступа в выводе ls -l}

  Вывод команды \alert{ls -l} содержит информацию о правах доступа:

  \lstinputlisting[basicstyle=\small,breaklines=false]{samples/ls-l-details}

  Позиции:
  \begin{itemize}
    \item 0 - тип  файла: - обычный;
    \item 1-3 - (\alert{u}) права доступа для владельца-пользователя.
    \item 4-6 - (\alert{g}) права доступа для владельца-группы.
    \item 7-9 - (\alert{o}) права доступа для остальных.
  \end{itemize}
\end{frame}

\begin{frame}{Значение прав доступа. Файлы и ссылки}

  \begin{block}{Обычные файлы}
    \begin{itemize}
      \item чтение \alert{(r)} надо, чтобы прочитать файл
	\footnote{Для успешного запуска скрипта необходимо установить атрибут \alert{r}, чтобы командный интерпретатор мог построчно считывать текст скрипта.}
      \item запись \alert{(w)}, чтобы файл изменить
      \item выполнение \alert{(x)}, чтобы запустить программу или скрипт.
    \end{itemize}
  \end{block} \pause

  \begin{block}{Символические ссылки}
    \small{Права символических ссылок совпадают с файлом, на который она указывает. На самой ссылке стоит 'всем всё разрешено'.}
  \end{block}

\end{frame}

\begin{frame}{Значение прав доступа. Каталоги}
  \begin{block}{Каталоги}
    \begin{itemize}
      \item \alert{(r)} позволяет получить имена (и только имена) файлов в нём\footnote{ls dir}. 
      \item \alert{(w)} TODO 
      \item \alert{(x)} позволяет "выполнить" каталог\newline
	то есть заглянуть в метаданные  и получить полную информацию о каталоге и файлах в нём\footnote{ls -l dir}.
    \end{itemize}
  \end{block}
\end{frame}

\section{Access Control Lists (ACL)}

\section{Поиск файлов}

\end{document}
