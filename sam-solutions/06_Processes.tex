% common part for every lection

\documentclass{beamer}

\usetheme{Warsaw}
\usefonttheme[onlylarge]{structurebold}
\setbeamerfont*{frametitle}{size=\normalsize,series=\bfseries}
\setbeamertemplate{navigation symbols}{}

\usepackage{pdfpages}
\usepackage{soul}
\usepackage{ucs}

% encoding settings
\usepackage[T2A]{fontenc}
\usepackage{xltxtra}
\usepackage{hyperref}
\usepackage{polyglossia}
\usepackage{xecyr}

\setdefaultlanguage{russian}

% fonts settings
\usepackage{fontspec}
\setmainfont[Ligatures=TeX]{CMU Serif}                 % Computer Modern Unicode font
\setsansfont[Ligatures=TeX]{CMU Sans Serif}
\setmonofont{CMU Typewriter Text}

\usepackage{listings}
\lstset{frame=single,breaklines=true,language=sh}

% for strike-out text
\usepackage[normalem]{ulem}

% tabulation aka \tbs
\newcommand{\tbs}{\tt\char`\\}

% \regex for regular expressions
\newcommand{\regex}[1]{ %
\expandafter{$\ulcorner{\color{blue}\texttt{#1}}\lrcorner$} %
}

% \qatrainingtitle
\newcommand{\qatrainingtitle}[2] {%
\title[SaM Solutions. Linux QA Training] %
{ %
  Часть #1.\\%
  #2 %
}

}

\newcommand{\firstframe}{ %
\begin{frame} %
  \titlepage %
\end{frame} %
}

\author[Author, Vlad Shakhov]{Влад 'mend0za' Шахов\\Linux \& Embedded Team Leader}

\institute[SaM Solutions]
{
  Linux \& Embedded Department
}

\date[Dec 2012]

\subject{Linux QA training}

\pgfdeclareimage[height=1.5cm]{sam-solutions-logo}{clipart/sam-solutions-elinux}

\logo{\pgfuseimage{sam-solutions-logo}}

\graphicspath{{./clipart/}}

 

\qatrainingtitle{6}{Процессы и потоки} 

\lstset{basicstyle=\tiny}

\begin{document}

\firstframe

\section{Процессы}

\begin{frame}{Процессы. Общая информация}
  \begin{block}{Процесс}
    \begin{itemize}
      \item Программа запускает 1 или более процессов. 
      \item Процесс состоит из инструкций, выполняемых процессором, данных и информации о выполняемой задаче (данные в памяти и стеке, открытые файлы и статус процесса).
    \end{itemize} 
  \end{block} \pause

  \begin{block}{Изоляция процессов}
    \begin{itemize}
      \item Процессы изолированы друг от друга\footnote{нет доступа к памяти и стеку, открытым файлам и порядку выполения}.
      \item Процессы могут обмениваться данными через систему межпроцессного взаимодействия (IPC)
      \item Следствие: пользователи могут запускать несколько экземпляров одной и той же программы. 
    \end{itemize}
  \end{block}

\end{frame}

\begin{frame}{Файлы и процессы. Права доступа}

  \begin{block}{Связь с файлами}
    Процессы запускаются из файлов (двоичных программ или скриптов).
  \end{block}

  \begin{block}{Права доступа}
    Процессы работают с правами пользователя, их запустившего\footnote{Если не установлены SUID или SGID биты на файле программы}
  \end{block}

\end{frame}

\begin{frame}{Виды процессов}
% TODO картинку с демоном
  \begin{block}{Демоны}
    Неинтерактивные процессы, выполняются в фоновом режиме. Не связаны ни с одним пользовательским сеансом и не могут непосредственно управляться пользователем. \newline
    Примеры: \alert{sshd}, \alert{apache}, \alert{cron}, \alert{samba}
  \end{block} \pause

  \begin{block}{Системные процессы}
    Часть ядра и всегда расположены в оперативной памяти. \newline
    Примеры: диспечер подкачки памяти ( \alert{[kswapd0]} )
  \end{block} \pause

  \begin{block}{Прикладные процессы}
    все остальные процессы. Запускаются в рамках пользовательского сеанса. \newline
    Примеры: \alert{ls}, \alert{bash}, \alert{vim}, \alert{find}, \alert{mysql}

  \end{block}

\end{frame}

\begin{frame}[fragile]{Практика: утилиты просмотр процессов}
    
  \begin{itemize}
      \item \alert{ps} - список запущенных процессов.\newline
	Популярные ключи: \alert{-u username}, 
	\alert{ax}\footnote{ \alert{ax} для BSD, \alert{-e} в SystemV}, 
	\alert{aux}\footnote{ \alert{aux} для BSD, \alert{-ef} в SystemV}
\lstinputlisting{samples/ps-ax}
      \item \alert{top} - интерактивный список процессов
\lstinputlisting{samples/top}
    \end{itemize}

\end{frame}

\begin{frame}{Состояния процесса}
  % TODO
\end{frame}

\begin{frame}{Виды межпроцессного взаимодействия (IPC)}

  \begin{block}{Пользовательские IPC}
    \begin{enumerate}
      \item \alert{файлы}
      \item \alert{каналы (трубы)} - именованные (FIFO) и неименованные (в Shell)\pause
      \item \alert{сигналы} - уведомление о возникновении события
    \end{enumerate}
  \end{block}\pause

  \begin{block}{IPC Для программистов}
    \begin{enumerate}
      \item \alert{разделяемая память}
      \item \alert{семафоры}
      \item \alert{очереди сообщений}
    \end{enumerate}
  \end{block}

\end{frame}

\section{Сигналы}


\begin{frame}{Сигналы}
  \begin{block}{Сигнал}
    \begin{enumerate}
      \item Cпособ передачи уведомления о возникновении какого-либо события.
      \item Сигнал может идти от одного процесса другому или от ядра ОС какому-либо процессу.
      \item Номер сигнала - единственная информация, которую он передаёт.
    \end{enumerate}
  \end{block} \pause

  \begin{block}{Права доступа}
    \begin{itemize}
      \item Пользователь может посылать сигналы только тем процессам, владельцем которых он является.
      \item \alert{root} может посылать сигналы любому процессу.
    \end{itemize}
  \end{block}

\end{frame}

\begin{frame}[fragile]{Работа с сигналами}
  \small Информация о сигналах: \alert{man 7 signal}, \alert{kill -l}
  \begin{block}{Популярные сигналы}
    \tiny
    \begin{tabular}{l|c|c|l}
      Сигнал & Значение & Действие
      .умолч.& Комментарий \\ \hline 
      SIGHUP &    1     &  Term   &  Обрыв соед. терминала или смерть упр. процесса \\
      SIGINT &    2     &  Term   & Прерывание с клавиатуры (Ctrl+C) \\
      SIGKILL &   9     &  Term   & Убить процесс \\
      SIGSEGV &   11    &  Core   & Segmentation fault \\
      SIGPIPE &   13    &  Term   & Broken pipe: запись в канал без читателей\\
      SIGTERM &   15    &  Term   & Прекратить процесс \\
      SIGCONT &   18    &  Cont   & Продолжить выполнение \\
      SIGSTOP &   19    &  Stop   & Остановить 
    \end{tabular}
  \end{block} \pause

  \begin{block}{Утилиты управления}
    \begin{itemize}
      \item \alert{kill} - послать сигнал процессу (по PID)
      \item \alert{killall} - послать сигнал процессам по имени
    \end{itemize}
  \end{block}

\end{frame}

\section{Shell и фоновые задания}

\section{Init-система}

\end{document}
