\begin{frame}
	\frametitle{Задача бизнеса}

	\begin{block}{Заработать денег владельцам}

	В сфере встраиваемых и серверных решений под управлением ОС Linux:
	\begin{itemize}
	  \item Сегмент Linux стремительно растёт
	  \item Проекты есть, людей нет
	  \item Люди не знают инструментальную среду Linux
	\end{itemize} 

      \end{block}
\end{frame}



\begin{frame}
	\frametitle{Почему нас так мало?}

	\begin{block}{Проблемы с изучением Linux в ВУЗах}
		\begin{itemize}
			\item Де-факто закрытый стек технологий на основе ОС Windows
			\item Отсутствует изучение профессиональных инструментов
			\item Нет практики совместной разработки
			\item Преподавателями игнорируются подходы, принятые в мире связанном со Свободным ПО
			\item Само наличие Linux в образовательном процессе является 
				заслугой исключительно отдельных лиц, работающих в ВУЗе
		\end{itemize}
	\end{block}
\end{frame}
  

\begin{frame}{Почему нас так мало?}
  \begin{block}{Сообщество. Открытое ли?}
    \begin{itemize}
      \item Кастовость
      \item Снобизм и псевдо 'элитарность'
      \item Сложность первоначального вхождения
    \end{itemize} 
  \end{block} \pause

  \alert{Т.е. те же болячки, что и у белорусского IT в целом}
\end{frame}


\begin{frame}
	\frametitle{Где взять Linux-специалистов?}

	\begin{block}{Источники}
		\begin{itemize}
			\item Система образования в РБ? \\
				LOL
				\pause
			\item Существующие специалисты? \\
				Циркуляция одних и тех же лиц.
				\pause
			\item Естественный приток энтузиастов? \\
				Слишком медленно.
		\end{itemize}
	\end{block}

	\begin{block}{Решение}
		\begin{itemize}
			\item Самостоятельная планомерная подготовка специалистов
			\item Создание благоприятной экосистемы для самозарождения
		\end{itemize}

	\end{block}
\end{frame}

