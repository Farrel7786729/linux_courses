\begin{frame}
	\frametitle{Задача бизнеса}

	\begin{block}{Заработать денег владельцам}

	Для решения этой задачи в сфере встраиваемых и серверных решений под управлением Linux 
	необходимы специалисты знающие эту операщионную систему.
	\end{block}
\end{frame}



\begin{frame}
	\frametitle{Почему нас так мало?}

	\begin{block}{Проблемы с изучением Linux в ВУЗах}
		\begin{itemize}
			\item Де-факто закрытый стек технологий на основе ОС Windows
			\item Отсутствует изучение профессиональных инструментов
			\item Нет практики совместной разработки
			\item Преподавателями игнорируются подходы, принятые в мире связанном со Свободным ПО
			\item Само наличие Linux в образовательном процессе является 
				заслугой исключительно отдельных лиц, работающих в ВУЗе
		\end{itemize}
	\end{block}

	\pause

	\begin{block}{Сообщество. Открытое ли?}
		\begin{itemize}
			\item Кастовость
			\item Снобизм и псевдо ''элитарность``
			\item Сложность первоначального вхождения
		\end{itemize}
	\end{block}
\end{frame}


\begin{frame}
	\frametitle{Где взять Linux-специалистов?}

	\begin{block}{}
		\begin{itemize}
			\item Система образования в РБ? 
				LOL
				\pause
			\item Существующие специалисты?
				Циркуляция одних и тех же лиц.
				\pause
			\item Естественный приток энтузиастов?
				Слишком медленно.
		\end{itemize}
	\end{block}

	\begin{block}{Решение}
		\begin{itemize}
			\item Самостоятельная планомерная подготовка специалистов
			\item Создание благоприятной экосистемы для самозарождения
		\end{itemize}

	\end{block}
\end{frame}

