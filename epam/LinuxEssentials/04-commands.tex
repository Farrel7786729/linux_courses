\documentclass[ignorenonframetext, professionalfonts, hyperref={pdftex, unicode}]{beamer}

\usetheme{Copenhagen}
\usecolortheme{wolverine}

%Packages to be included
%\usepackage{graphicx}

\usepackage[russian]{babel}
\usepackage[utf8]{inputenc}
\usepackage[T1]{fontenc}

%%\usepackage[orientation=landscape, size=custom, width=16, height=9.75, scale=0.5]{beamerposter}

\usepackage{textcomp}

\usepackage{beamerthemesplit}

\usepackage{ulem}

\usepackage{verbatim}

\usepackage{ucs}


\usepackage{listings}
\lstloadlanguages{bash}

\lstset{escapechar=`,
	extendedchars=false,
	language=sh,
	frame=single,
	tabsize=2, 
	columns=fullflexible, 
%	basicstyle=\scriptsize,
	keywordstyle=\color{blue}, 
	commentstyle=\itshape\color{brown},
%	identifierstyle=\ttfamily, 
	stringstyle=\mdseries\color{green}, 
	showstringspaces=false, 
	numbers=left, 
	numberstyle=\tiny, 
	breaklines=true, 
	inputencoding=utf8,
	keepspaces=true,
	morekeywords={u\_short, u\_char, u\_long, in\_addr}
	}

\definecolor{darkgreen}{cmyk}{0.7, 0, 1, 0.5}

\lstdefinelanguage{diff}
{
    morekeywords={+, -},
    sensitive=false,
    morecomment=[l]{//},
    morecomment=[s]{/*}{*/},
    morecomment=[l][\color{darkgreen}]{+},
    morecomment=[l][\color{red}]{-},
    morestring=[b]",
}

\author[Epam]{{\bf Epam}\\Low Level Programming Department}

%\institution[EPAM]{EPAM}
%\logo{\includegraphics[width=1cm]{logo.png}}


\title{Введение в GNU/Linux}


%%%%%%%%%%%%%%%%%%%%%%%%%%%%%%%%%%%%%%%%%%%%%%%%%
%%%%%%%%%% Begin Document  %%%%%%%%%%%%%%%%%%%%%%
%%%%%%%%%%%%%%%%%%%%%%%%%%%%%%%%%%%%%%%%%%%%%%%%%




\begin{document}

\begin{frame}
	\frametitle{}
	\titlepage
	\vspace{-0.5cm}
	\begin{center}
	%\frontpagelogo
	\end{center}
\end{frame}


%%%%%%%%%%%%%%%%%%%%%%%%%%%%%%%%%%%%%%%%%   
%%%%%%%%%% Content starts here %%%%%%%%%%
%%%%%%%%%%%%%%%%%%%%%%%%%%%%%%%%%%%%%%%%%

\section{Полезные команды (продолжение)}

\subsection{Архиваторы}
\mode<all>{\begin{frame}[fragile]{Архивация}
	\begin{block}{Архивация: tar}
		\begin{itemize}
			\item {\tt -c} -- создать архив
			\item {\tt -x} -- извлечь из архива
				\begin{itemize}
					\item {\tt -C} -- перейти в директорию
					\item {\tt -{}-strip-components=N} -- пропустить N уровней
				\end{itemize}
			\item {\tt -f} -- запись в файл
		\end{itemize}
	\end{block}

	\begin{block}{Сжатие: gzip, bzip, xz}
		\begin{itemize}
			\item {\tt -[1-9]} -- изменить уровень сжатия
			\item {\tt -d} -- распаковать
			\item {\tt -c} -- вывод на консоль
		\end{itemize}
		\begin{verbatim}
dd if=/dev/sda bs=1M count=1 | gzip -c > backup.gz
		\end{verbatim}
	\end{block}

\end{frame}

\begin{frame}[fragile]{Архивация: примеры}

	Создать сжатый архив:
	\begin{verbatim}
tar -czf archive.tar.gz *
	\end{verbatim}
	\pause
	Распаковать сжатый архив в директорию {\tt /tmp}:
	\begin{verbatim}
tar -C /tmp/ -xzf archive.tar.gz 
	\end{verbatim}
	\pause
	Создать сжатый архив:
	\begin{verbatim}
tar -czf archive.tar.gz *
	\end{verbatim}
	\pause
	Создать копию текущей директории в директории {\tt /tmp/copy/}:
	\begin{verbatim}
tar -c * | tar -C /tmp/copy -x
tar -cf - * | tar -C /tmp -xf -
	\end{verbatim}
	\pause
	Создать копию текущей директории на другом хосте:
	\begin{verbatim}
HostDest: netcat -l 2222 | gzip -dc | tar -C /tmp/copy/ -x
HostSrc:  tar -c * | gzip -9 | netcat HostDest 2222
	\end{verbatim}
\end{frame}
}

\subsection{find и xargs}
\mode<all>{\begin{frame}[fragile]{Поиск файлов}
	\begin{block}{find}
		\begin{itemize}
			\item {\tt -type} -- тип файлового объекта
			\item {\tt -size} -- размер
			\item {\tt -maxdepth} -- глубина рекурсии
			\item {\tt -exec} -- выполнить команду
			\item {\tt -printf} -- форматированный вывод
		\end{itemize}
	\end{block}

	\begin{block}{Примеры}
		\begin{verbatim}
find /etc -type f -size +100k  -exec ls -l {} \;
		\end{verbatim}

		\begin{verbatim}
find -type d -user altlinux
		\end{verbatim}
	
	\end{block}
\end{frame}

\begin{frame}[fragile]{xargs}
	\begin{block}{xargs}
			Утилита для создания и запуска команд из стандартного потока ввода:
		\begin{verbatim}
xargs [options] command [command options]
		\end{verbatim}

		\begin{itemize}
			\item {\tt -d} -- разделитель
			\item {\tt -0} -- null-terminated строки
			\item {\tt -I text} -- подстановка
			\item {\tt -n N} -- максимальное количество аргументов
			\item {\tt -P N} -- максимальное количество процессов
		\end{itemize}

	\end{block}
\end{frame}

\begin{frame}[fragile]{xargs}
	\begin{block}{Примеры}
		\begin{verbatim}
find /etc -type f -size -100k | xargs tar -czf /tmp/archive-100k.tar.gz
		\end{verbatim}

		\begin{verbatim}
find /etc -type f | xargs -I {} echo "Найден {} файл"
		\end{verbatim}

		\begin{verbatim}
find . -type f -name "*.mp3" -print0 | xargs -0 -n 1 -P 0 -I mp3 avconv -i mp3 mp3.ogg
		\end{verbatim}
	
	\end{block}
\end{frame}


}

\subsection{Редакторы}
\mode<all>{\begin{frame}{Редакторы}
	\begin{itemize}
		\item Интерактивные
			\begin{itemize}
				\item vi
					\begin{itemize}
						\item Есть почти везде
					\end{itemize}
				\item vim
				\item emacs
			\end{itemize}
		\item Поточные
			\begin{itemize}
				\item {\tt ed}
				\item {\tt sed}
				\item {\tt awk}
			\end{itemize}
	\end{itemize}
\end{frame}

\begin{frame}[fragile]{Метасимволы}
	\begin{block}{grep, sed, awk}
	\end{block}
	\begin{itemize}
		\item {\tt .} -- любой символ за исключением пустой строки
		\item {\tt *} -- любоe количество символов, которые стоят перед {\tt *}
		\item {\tt \^{}} -- начало строки
		\item {\tt \$} -- конец строки
		\item {\tt [...]} -- любой символ из заключенных в скобки
	\end{itemize}
\end{frame}

\begin{frame}[fragile]{sed}
	\begin{block}{Сценарии}
		{\tt [ addr [ ,  addr ] ] cmd [ args ]}
	\end{block}

	\tiny
	\begin{block}{Команды}
		\begin{itemize}
		  \item {\tt a, i} -- добавить строку после (перед) текущей
			  \begin{verbatim} who | sed -e 'a Text' \end{verbatim}
		  \item {\tt c} -- удалить строку и заменить на текст
			  \begin{verbatim} who | sed -e "/$USER/ c Юзверь" \end{verbatim}
		  \item {\tt d} -- удалить строку
			  \begin{verbatim} who | sed -e '2,4 d' \end{verbatim}
			  \begin{verbatim} who | sed -e '/pts/ d' \end{verbatim}
		  \item {\tt s} -- замена по регулярному выражению
			  \begin{verbatim} who | sed -e "s/$USER/Юзверь/g" \end{verbatim}
		\end{itemize}
	\end{block}
\end{frame}


}

\section{Система управления пакетами}

\mode<all>{\begin{frame}
	\frametitle{И еще раз про "DLL hell"}
	
	\begin{block}{Устанавливаем программу}
	А что же с библиотеками?
	\end{block}

	\pause

	\begin{columns}
		\column{0.5\textwidth}
		\begin{block}{"В системе все есть!"}
		\begin{itemize}
			\item Oh, really???
			\item И нужной версии?
			\item А API и ABI точно не менялись?
			\item А если библиотек несколько версий?
			\item А если нужны дополнительные программы?
		\end{itemize}
		\end{block}
		\pause
		\column{0.5\textwidth}
		\begin{block}{"Всё своё, ношу с собой!"}
		\begin{itemize}
			\item А как насчет объема?
			\item Использование памяти.
			\item А что насчет лицензий?
			\item И все-таки порядок загрузки...
			\item Не спасает от проблем с 3rd-party ПО.
		\end{itemize}
		\end{block}
	\end{columns}
\end{frame}

\begin{frame}
	\frametitle{Хаос}

	\begin{center}
		"Даешь каждой платформе и языку собственную систему управления пакетами!"
	\end{center}

	\begin{block}{Увы, мы не в идеальном мире}
		\begin{itemize}
			\item Дистрибутивы: rpm\{4,5\}, deb, portage, pacman... и куча модификаций...
			\item Дополнительный софт: {\tt ./configure; make; make install}
			\item Java: {\tt ivy, ant, maven, gradle}
			\item Ruby: gem
			\item Perl: CPAN
			\item Python: pip + PyPi
		\end{itemize}
	\end{block}

\end{frame}

\begin{frame}
	\frametitle{Разработка и использование в реальной системе}
	
	\begin{block}{Build-time vs Run-time}

		\begin{enumerate}
			\item Не все, что нужно во время компиляции, должно быть установлено в конечной системе.
			\item Не все, что нужно для работы программы, необходимо устанавливать на сборочной системе.
		\end{enumerate}
	\end{block}
%TODO перенести после описания структуры каталогов
	\begin{block}{Чистое сборочное окружение}
		\begin{itemize}
			\item Воспроизводимость сборки 
			\item Контроль зависимостей
			\item Контроль автоматически "подхваченных" зависимостей
		\end{itemize}
	\end{block}
\end{frame}

}
\mode<all>{\begin{frame}{Система управления пакетами: для чего это нужно}
\begin{itemize}
 \item ''DLL Hell''
 \item Dependency hell
 \item Общие задачи пакетного менеджера:
   \begin{itemize}
     \item Проверка целостности пакетов
     \item Проверка зависимостей пакетов
        \item Поддержание списка установленных пакетов
        \item Автоматическое удаление пакетов
     \item Предоставление доступа к репозиторию пакетов
     \item Разрешение зависимостей
   \end{itemize}
\end{itemize}
\end{frame}

\begin{frame}{Debian-based и RedHat-based системы управления пакетами}
\begin{center}
 \textbf{Два уровня пакетных менеджеров}
\end{center}
\begin{columns}
  \column{0.4\textwidth}
  \begin{center}
    \textbf{RedHat-based}
  \end{center}
  \begin{itemize}
    \item dnf/yum
    \item rpm
  \end{itemize}
  \column{0.4\textwidth}
  \begin{center}
    \textbf{Debian-based}
  \end{center}
  \begin{itemize}
    \item aptitude, apt, synaptic
    \item dpkg
  \end{itemize}
\end{columns}
\end{frame}
}

\mode<all>{\begin{frame}{RPM: команды}
	\begin{block}{Установка пакета}
		{\tt rpm -i [rpm-file1] ... [[url://]rpm-fileN] }
	\end{block}
	\begin{block}{Удаление пакета}
		{\tt rpm -e pkgname1 ... pkgnameN }
	\end{block}
	\begin{block}{Обновление пакета}
		{\tt rpm -U [rpm-file1] ... [[url://]rpm-fileN] }
	\end{block}
	\begin{block}{Проверка пакета}
		{\tt rpm -V pkgname1 ... pkgnameN }
	\end{block}
\end{frame}

\begin{frame}{RPM: часто используемые опции опроса}

	\begin{itemize}
		\item {\tt pkgname} -- выбор пакета, установленного в системе
		\item {\tt -a} -- все пакеты, установленные в системе
		\item {\tt -p} -- использовать файл RPM
	\end{itemize}


	\begin{itemize}
		\item {\tt -i} -- показать информацию пакета\\
			{\tt rpm -q -i glibc }
		\item {\tt -l} -- показать список файлов пакета \\
			{\tt rpm -q -l glibc }
		\item {\tt --whatprovides} -- \\
			{\tt rpm -q --whatprovides java}
		\item {\tt --whatrequires} -- \\
			{\tt rpm -q --whatrequires /bin/bash}
		\item {\tt --queryformat} -- формат вывода\\
			{\tt rpm -q --whatrequires /bin/bash --queryformat ''\%\{name\} ''}

	\end{itemize}

\end{frame}


}

\mode<all>{\newcounter{tmpc}

\begin{frame}{Репозиторий}
	\begin{block}{Репозиторий пакетов}
		Место, где хранятся и поддерживаются пакеты, а также сопутствующая мета-информация, предназначенное для использования пакетным менеджером.
	\end{block}
	\begin{block}{Пример: Fedora Core}
		\begin{itemize}
			\item Packages/*.rpm
			\item RPM-GPG-KEY-*
			\item repodata
			\begin{itemize}
				\item множество сжатых и несжатых XML файлов для YUM
			\end{itemize}
		\end{itemize}

		Описание репозтория для YUM на локальной системе хранится по пути
		{\tt /etc/yum.repos.d/*.repo}
	\end{block}
		
\end{frame}

\begin{frame}{YUM: команды}
	\begin{block}{Установка/обновление пакета}
		{\tt yum install pkgname }
	\end{block}
	\begin{block}{Обновление всех пакетов}
		{\tt yum update }
	\end{block}
	\begin{block}{Удаление пакета}
		{\tt yum remove pkgname }
	\end{block}
	\begin{block}{Поиск}
		{\tt yum list pkgname }\\
		{\tt yum search pkgname }
	\end{block}
\end{frame}


\begin{frame}[fragile]{Упражнение}
  \begin{enumerate}
      \item Создать на {\tt /dev/sda} раздел размером примерно 10Gb
      \item Создать на этом разделе ext3 ФС и смонтировать раздел в {\tt /mnt/chroot}
      \item Развернуть {\tt /media/nfs/pub/CentOS/precreated/centOS.tar.gz} в {\tt /mnt/chroot}
      \item Смонтировать {\tt proc, sysfs} а также {\tt /dev} в соответствующие места {\tt /mnt/chroot}
      \item {\tt chroot /mnt/chroot}
      \item Отредактировать {\tt /etc/resolv.conf} -- скопировать туда информацию из {\tt resolv.conf} основной системы
      \item Отредактировать {\tt /etc/yum.conf} Добавить следующий раздел
\begin{minipage}{0.5\textwidth}
\begin{verbatim}
[base]
  name = CentOS 6
  baseurl = ftp://192.168.11.15/CentOS
  gpgcheck = 0
\end{verbatim}
\end{minipage}
\setcounter{tmpc}{\theenumi}
\end{enumerate}
\end{frame}
\begin{frame}{Продолжение упражнения}
  \begin{enumerate}
      \setcounter{enumi}{\thetmpc}
      \item {\tt yum update}
      \item Установить пакет vim
      \item Посмотреть списки файлов для пакетов {\tt yum, rpm}
      \item Найти пакет предоставляющий сервис ssh и установить его
      \item Удалить пакет vim
    \end{enumerate}
\end{frame}


}

\end{document}
