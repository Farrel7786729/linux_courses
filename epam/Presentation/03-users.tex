\documentclass[ignorenonframetext, professionalfonts, hyperref={pdftex, unicode}]{beamer}

\usetheme{Copenhagen}
\usecolortheme{wolverine}

%Packages to be included
%\usepackage{graphicx}

\usepackage[russian]{babel}
\usepackage[utf8]{inputenc}
\usepackage[T1]{fontenc}

%%\usepackage[orientation=landscape, size=custom, width=16, height=9.75, scale=0.5]{beamerposter}

\usepackage{textcomp}

\usepackage{beamerthemesplit}

\usepackage{ulem}

\usepackage{verbatim}

\usepackage{ucs}


\usepackage{listings}
\lstloadlanguages{bash}

\lstset{escapechar=`,
	extendedchars=false,
	language=sh,
	frame=single,
	tabsize=2, 
	columns=fullflexible, 
%	basicstyle=\scriptsize,
	keywordstyle=\color{blue}, 
	commentstyle=\itshape\color{brown},
%	identifierstyle=\ttfamily, 
	stringstyle=\mdseries\color{green}, 
	showstringspaces=false, 
	numbers=left, 
	numberstyle=\tiny, 
	breaklines=true, 
	inputencoding=utf8,
	keepspaces=true,
	morekeywords={u\_short, u\_char, u\_long, in\_addr}
	}

\definecolor{darkgreen}{cmyk}{0.7, 0, 1, 0.5}

\lstdefinelanguage{diff}
{
    morekeywords={+, -},
    sensitive=false,
    morecomment=[l]{//},
    morecomment=[s]{/*}{*/},
    morecomment=[l][\color{darkgreen}]{+},
    morecomment=[l][\color{red}]{-},
    morestring=[b]",
}

\author[Epam]{{\bf Epam}\\Low Level Programming Department}

%\institution[EPAM]{EPAM}
%\logo{\includegraphics[width=1cm]{logo.png}}

\title{Введение в GNU/Linux}


%%%%%%%%%%%%%%%%%%%%%%%%%%%%%%%%%%%%%%%%%%%%%%%%%
%%%%%%%%%% Begin Document  %%%%%%%%%%%%%%%%%%%%%%
%%%%%%%%%%%%%%%%%%%%%%%%%%%%%%%%%%%%%%%%%%%%%%%%%




\begin{document}

\frame{
	\frametitle{}
	\titlepage
	\vspace{-0.5cm}
	\begin{center}
	%\frontpagelogo
	\end{center}
}
\frame{
	\tableofcontents
%	[hideallsubsections]
}

\begin{frame}{Практика: работа с файловыми объектами}
	\begin{enumerate}
		\item Cкопировать файл1 в файл2
			\pause
		\item Добиться того, чтобы файл2 состоял из 3 одинаковых копий файла1
		\item Создать жесткую ссылку на файл1
		\item Создать символическую ссылку на файл2
		\item Вывести на экран список всех файлов
			\pause
		\item Добавить вывод команды {\tt date} в файл1
		\item Вывести на экран список всех файлов
		\item Вывести на экран содержимое файла1 и жесткой ссылки
		\item Создать именованный канал (PIPE)
		\item Запустить одну команду {\tt cat} на чтение из PIPE
		\item Записать содержимое файла1 и файла2 в PIPE
		\item Удалить временную директорию
	\end{enumerate}
\end{frame}
\begin{frame}{Многопользовательская модель}   
 \begin{itemize}
   \item Linux -- многопользовательская система
   \item Привилегии пользователей
     \begin{itemize}
       \item root
       \item other users
      \end{itemize}
     \end{itemize}
\end{frame}
\section{Механизмы разделения привилегий}
\subsection{Классический UNIX}
\begin{frame}{Пользователи, группы и файлы}
\begin{itemize}
  \item Каждый пользователь принадлежит одной или нескольким \textbf{группам}
  \item Каждый файл и директория принадлежит
    \begin{itemize}
      \item Одному пользователю 
      \item Одной группе
    \end{itemize}
  \pause
  \item  Разрешения что либо делать с файлом определяются по отношению к
    \begin{enumerate}
      \item Пользователю-владельцу файла
      \item Группе владеющей файлом
      \item Всем остальным пользователям
    \end{enumerate}

\end{itemize}
\pause
\begin{columns}
  \column{0.48\textwidth}
  \begin{itemize}
    \item {\tt ls -l} 3,4 поле 
    \item {\tt groups}
   \end{itemize}
  \column{0.48\textwidth}
  \begin{block}{Попробовать}
    {\tt ls -l /usr/bin/}

    {\tt groups}

    {\tt groups root}
  \end{block}
\end{columns}
\end{frame}

\begin{frame}{Типы разрешений для файлов}
\begin{columns}
\column{0.48\textwidth}
\begin{center}
  \textbf{Разрешения для файла}
\end{center}
\begin{itemize}
  \item Три типа разрешений
    \begin{enumerate}
      \item чтение read(r)
      \item запись write(w)
      \item выполнение execute(x)
    \end{enumerate}
\end{itemize}
\column{0.48\textwidth}
\begin{center}
  \textbf{Разрешения для директорий}
\end{center}
\begin{itemize}
  \item Три типа разрешений
    \begin{enumerate}
      \item поиск файлов в директории read(r) 
      \item добавление и удаление файлов write(w)
      \item заход в директорию execute(x)
    \end{enumerate}
\end{itemize}
\end{columns}

\pause

Попробовать {\tt ls -l /usr/bin}

\pause

Пересчет мнемонического разрешения в битовую маску 

$r\to4, w\to2 , x\to1$ 

rwxrw-r-x$\to$765
\end{frame}

\begin{frame}{Команды для управления пользователями и разрешениями файлов}
\begin{columns}
 \column{0.48\textwidth}
 \begin{itemize}
   \item {\tt chown}
   \item {\tt chmod}
 \end{itemize}
 \column{0.48\textwidth}
 \begin{itemize}
   \item {\tt useradd, usermod, userdel}
   \item {\tt groupadd, groupmod, groupdel}
   \item {\tt su, sudo}
 \end{itemize}
\end{columns}
\end{frame}

\begin{frame}
\begin{block}{Упражнения}
  \begin{enumerate}
   \item Создать директорию без r разрешения но с x разрешением, внутри нее создать поддиректорию с rwx разрешениями (для пользователя altlinux)
   \item Создать нового пользователя testuser.
   \item Скопировать {\tt /bin/bash} (под именем mysh) в домашнюю директорию пользователя altlinux  и поставить r-x разрешение только для other
   \item Попробовать выполнить скопированный файл от имени пользователя altlinux, затем от имени пользователя testuser
   \item Создать новую группу testgroup
   \item Изменить группу владеющую mysh на testgroup и сделать {\tt chmod 464 mysh}
   \item Попробовать выполнить mysh от имени altlinux и root. 
   \item Добавить пользователя altlinux в группу testgroup и попробовать выполнить mysh еще раз
   \item Получить список групп которым принадлежат устройства в {\tt /dev}
  \end{enumerate}
\end{block}
\end{frame}

\begin{frame}{SUID программы}
 \begin{block}{Попробовать}
   {\tt id}

   {\tt ls -l `which su`}
 \end{block}
 \pause
 \begin{itemize}
   \item Некоторые программы должны выполняться от имени обычного пользователя, но иметь больше привилегий
   \item Для этого у них устанавливается suid или sgid биты
   \item Установка suid (например {\tt chmod 4710 <file>})
 \end{itemize}
 \pause
 \begin{block}{Упражнение}
   \begin{itemize}
     \item Под root создать копию утилиты {\tt id} (назвать, например, {\tt id2}) в директории /usr/bin/
     \item Установить suid бит для этой утилиты
     \item Запустить {\tt id2} от имени пользователя altlinux
     \item То же с sgid битом
    \end{itemize}
 \end{block}
\end{frame}

\begin{frame}{Опасности SUID}
  \begin{itemize}
    \item Возможность backdoor через suid программу
      \begin{itemize}
        \item Shell игнорирует effective uid
        \item Скрипты обычно тоже игнорируют
        \item nosuid mount option
       \end{itemize}
     \item Атака через buffer overflow в существующей suid программе
       \begin{itemize}
         \item не использовать strcpy, sprintf, ... в security critical
         \item А если все же не уследили
           \begin{itemize}
             \item рандомизация стека
             \item grsecurity
             \item частично selinux
           \end{itemize}
       \end{itemize}
   \end{itemize}
\end{frame}


\begin{frame}{SUID, SGID и sticky bit для директорий}
  \begin{itemize}
    \item sgid для директорий -- все поддиректории и файлы внутри имеют тот же group id
    \item suid -- игнорируется
    \item Зато есть sticky bit 
   \end{itemize}
\end{frame}

% End of 5-th lecture

\begin{frame}{Хранение информации о пользователях в системе}

	\begin{block}{\tt /etc/group}
		{\tt  group\_name:password:GID:user\_list}
	\end{block}
	
	\pause

	\begin{block}{\tt /etc/passwd}
		{\tt account:password:UID:GID:GECOS:directory:shell}

		\begin{itemize}
			\item {\tt *} -- пароль не задан
			\item {\tt x} -- пароль задан в файле {\tt /etc/shadows}
		\end{itemize}
	\end{block}

	\pause

	\begin{block}{\tt /etc/shadow}
		\begin{enumerate}
			\begin{columns}
			\column{0.3\textwidth}

			\item login name
			\item encrypted password
			\item date of last password change

			\column{0.3\textwidth}		
			\item minimum password age 
			\item maximum password age
			\item password warning period

			\column{0.3\textwidth}
			\item password inactivity period
			\item account expiration date
			\item reserved field

			\end{columns}
		\end{enumerate}
	\end{block}

\end{frame}

\begin{frame}{Практическое задание}
    \begin{itemize}
		\item Посмотреть права доступа к файлам {\tt group}, {\tt passwd}, {\tt shadow}\\
			{\tt ls -l /etc/{group, passwd, shadow}}
		\item Добавить пользователя и группу и посмотреть изменения в перечисленных файлах
		\item Cоздать пользователя без использования системных утилит
    \end{itemize}
	\pause
	 \begin{itemize}
		\item Изменить пароль пользователю с помощью утилиты {\tt passwd}\\
			Hint: {\tt /etc/passwdqc.conf}
		\item Сбросить пароль пользователю\\
			Hint: {\tt usermod}
    \end{itemize}
\end{frame}


\begin{frame}{PAM}
	% http://www.opennet.ru/base/net/pam_linux.txt.html
	\begin{itemize}
		\item PAM это динамическая библиотека
		\item Конфигурация PAM
			\begin{itemize}
				\item {\tt /etc/pam.conf}
				\item {\tt /etc/pam.d/...}
					\begin{itemize}
						\item Сервисы
						\item system\_auth
					\end{itemize}
			\end{itemize}
	\end{itemize}

	\begin{block}{Формат записи}
		\begin{columns}
			\column{0.245\textwidth}
			\textbf{module type}
			 \begin{itemize}
				 \item auth
				 \item account
				 \item session
				 \item password
			 \end{itemize}
			 \column{0.245\textwidth}
			 \textbf{control flag}
			 \begin{itemize}
				 \item requisite
				 \item required
				 \item sufficient
				 \item optional
			 \end{itemize}
			 \column{0.245\textwidth}
			 \textbf{module name}
			 \column{0.245\textwidth}
			 \textbf{module options}
		 \end{columns}
	 \end{block}
\end{frame}

\begin{frame}{Диспетчер службы имен (NSS)}

	Важная информация для системы:
		
	\begin{itemize}
		\item Информация о пользователях (логин, группа, пароль и т.д.)
		\item Информация о сетевых ресурсах (имена хостов, протоколов, сервисов)
    \end{itemize}

	\pause

	\begin{block}{NSS}
		\begin{itemize}
			\item Конфигурация: {\tt /etc/nsswitch.conf}
			\item Динамические библиотеки сервисов: {\tt ls -1 /lib*/libnss\_*}
		\end{itemize}
	\end{block}

\end{frame}



\end{document}
