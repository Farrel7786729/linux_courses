\documentclass[ignorenonframetext, professionalfonts, hyperref={pdftex, unicode}]{beamer}

\usetheme{Copenhagen}
\usecolortheme{wolverine}


\title[bash]{Bourne again shell}
\author[Epam]{{\bf Epam}\\Embedded solutions department}

%Packages to be included

\usepackage[russian]{babel}
\usepackage[utf8]{inputenc}
\usepackage[T1]{fontenc}


\usepackage[orientation=landscape, size=custom, width=16, height=9.75, scale=0.5]{beamerposter}


\usepackage{textcomp}


\usepackage{beamerthemesplit}

\usepackage{ulem}

\usepackage{verbatim}

\usepackage{ucs}


\usepackage{listings}
\lstloadlanguages{bash}

\lstset{escapechar=`,
	extendedchars=false,
	language=C, 
	tabsize=2, 
	columns=fullflexible, 
%	basicstyle=\scriptsize,
	keywordstyle=\color{blue}, 
	commentstyle=\itshape\color{brown},
%	identifierstyle=\ttfamily, 
	stringstyle=\mdseries\color{green}, 
	showstringspaces=false, 
	numbers=left, 
	numberstyle=\tiny, 
	breaklines=true, 
	inputencoding=utf8,
	keepspaces=true,
	morekeywords={u\_short, u\_char, u\_long, in\_addr}
	}

\definecolor{darkgreen}{cmyk}{0.7, 0, 1, 0.5}

\lstdefinelanguage{diff}
{
    morekeywords={+, -},
    sensitive=false,
    morecomment=[l]{//},
    morecomment=[s]{/*}{*/},
    morecomment=[l][\color{darkgreen}]{+},
    morecomment=[l][\color{red}]{-},
    morestring=[b]",
}




\begin{document}





%%%%%%%%%%%%%%%%%%%%%%%%%%%%%%%%%%%%%%%%%%%%%%%%%
%%%%%%%%%% Begin Document  %%%%%%%%%%%%%%%%%%%%%%
%%%%%%%%%%%%%%%%%%%%%%%%%%%%%%%%%%%%%%%%%%%%%%%%%



\begin{frame}
	\frametitle{BASH}
	\titlepage
	\vspace{-0.5cm}
	\begin{center}
	%\frontpagelogo
	\end{center}
\end{frame}

\begin{frame}
	\tableofcontents
%	[hideallsubsections]
\end{frame}



%%%%%%%%%%%%%%%%%%%%%%%%%%%%%%%%%%%%%%%%%   
%%%%%%%%%% Content starts here %%%%%%%%%%
%%%%%%%%%%%%%%%%%%%%%%%%%%%%%%%%%%%%%%%%%

\section{Тесты и сравнения}

\mode<all>{%%

\begin{frame}{Общий синтаксис}

	Условие:
	\begin{itemize}
		\item Exit status любой программы
		\item test или $[$ 
		\item Двойные скобки {\bf (( ... ))} и конструкция {\bf let}
	\end{itemize}


	Конструкция для сравнения:
	\begin{itemize}
		\item \&\& и ||
		\item if/then/else
	\end{itemize}

\end{frame}

\begin{frame}[fragile]{Двойные скобки и {\bf let}}

	Позволяют производить арифметические вычисления!

	\pause
	Пример:
\begin{lstlisting}[language=bash]
a=$(( 5 + 3 ))
let "b = (( a + 10 ))"
echo $a $b
\end{lstlisting}

	\pause
	Операции:
	\begin{itemize}
		\item (( 0 \&\& 1 )) -- логическое "И"
		\item (( 0 || 1 )) -- логическое "ИЛИ"
		\item (( 0 \& 1 )) -- побитовое "И"
		\item (( 0 | 1 )) -- побитовое "ИЛИ"
	\end{itemize}

	{\bf Практическое задание:} \\
	написать скрипт, которому в качестве параметров передается 2 значения. Вывести на экран результаты логических и побитовых операций. 
\end{frame}


\begin{frame}[fragile]{test}

	\begin{itemize}
	    \item ! -- отрицание
	    \item -z СТРОКА
	    \item СТРОКА1 = СТРОКА2
	    \item СТРОКА1 != СТРОКА2
	    \item ЦЕЛОЕ1 -eq ЦЕЛОЕ2
	    \item ЦЕЛОЕ1 -ge ЦЕЛОЕ2
	    \item ЦЕЛОЕ1 -lt ЦЕЛОЕ2
	    \item -d ФАЙЛ
	    \item -e ФАЙЛ
	    \item -f ФАЙЛ
	\end{itemize}

\end{frame}

\begin{frame}[fragile]{\&\& и ||}
	Синтаксис:
\begin{verbatim}
условие && true || false
\end{verbatim}

	\pause
	Пример:
\begin{lstlisting}[language=bash]
test -z "$DISPLAY" && echo "text mode" || echo "graphical mode"
\end{lstlisting}
	
	\pause

	\begin{itemize}
	    \item Запустить команды "true" и "false"
	    \item В случае успеха вывести "успех"
	    \item В случае неуспеха вывести "неудача"
	\end{itemize}
\end{frame}



\begin{frame}[fragile]{Синтаксис {\bf if}}

	\begin{columns}
		\column{0.4\textwidth}
	
	\begin{lstlisting}[language=bash]
if [ условие1 ]
then
   . . .
elif [ условие2 ]
then
   . . .
else
   . . .
fi
\end{lstlisting}
		\column{0.6\textwidth}
	{\bf Практическое задание:} \\
	\begin{itemize}

		\item с помощью конструкции {\bf if} проверить существует ли файловый объект передаваемый в качестве параметра скрипту
		\item если нет, то создать директорию с таким именем
		\item если cуществует и файл является shell-скриптом, то запустить его
		\item если существует и является директорией, то вывести на экран первых 5 файлов в этой директории
	\end{itemize}
	\end{columns}
\end{frame}


%\begin{frame}[fragile]{}

%\end{frame}

}
%\section{Операторы}

%\mode<all>{%%


\begin{frame}{}

\end{frame}
}

\section{Арифметические операции}

\mode<all>{%% Arithmetic

\begin{frame}
  \frametitle{}
  \begin{itemize}
   \item  Конструкция {\tt ((...))}
    \begin{block}{Примеры}
     {\tt (( a=10 )); echo \$(( a++ )); echo \$a; } 
    \end{block}
    \pause
   \item  {\tt let}
    \begin{block}{Примеры}
     {\tt let a=10; echo \$a; let a+=-2; echo \$a; echo \$(( $--$a)); echo \$a} 
    \end{block}
    \pause
   \item  {\tt expr } внешняя команда 
    \begin{block}{Пример}
       { \tt x=\`{}expr \$x + 1\`{} }
    \end{block}
  \end{itemize}
\end{frame}

\begin{frame}[fragile]
\frametitle{Операторы в арифметических выражениях}
\begin{enumerate}
\item Инкременты, декременты {\tt id++, id$--$, ++id, $--$id } 
\item Арифметические операторы {\tt **,*,/,\%,+,-} 
\item Побитовые операторы {\verb+ ~,>>,<<,^,&,|+}
\item Операторы сравнения {\tt <=,>=,<,>, ==, !=}
\item Логические операторы {\tt \&\&, || } 
\item Тернарный оператор {\tt expr ? expr : expr }
\item Операторы присваивания
\begin{lstlisting}[language=C]
=, *=, /=, %=,
+=, -=, <<=, >>=,
&=, ^=, |=  
\end{lstlisting}
\end{enumerate} 

  
\end{frame}


\begin{frame}[fragile]
  \frametitle{Упражнение}
 \begin{enumerate}
   \item {\bf Конец света:} Вывести дату конца юниксовых времен ( время $2^{31}-1$) {\tt date -d @<seconds> } 
   \item Проверить результат арифметической операции (( 5>10 ))
 \end{enumerate}
\end{frame}
}

\section{Циклы}

\mode<all>{\begin{frame}
\frametitle{Основные конструкции для циклов}
  \begin{itemize}
   \item while
   \item for
   \item until
   \item break, continue 
   \item внешние команды find, xargs 
  \end{itemize}
\end{frame}

\begin{frame}[fragile]
  \frametitle{Циклы for}
  \begin{enumerate}
    \item Стандартная форма
\begin{lstlisting}[language=sh,frame=single]
  for x in list 
  do
    op1
    op2
  done
\end{lstlisting}
    \item Арифметическая форма
\begin{lstlisting}[language=sh,frame=single]
  for (( expr1 ; expr2 ; expr3 )) 
  do 
    op1
    op2
  done
\end{lstlisting}
  \end{enumerate}
\end{frame}

\begin{frame}[fragile]
\frametitle{ Циклы for. Примеры.}
  \begin{block}{Действие над файлами.}
\begin{lstlisting}[language=sh,frame=single]
for file in *
 do md5sum $file
done
\end{lstlisting}
  \end{block}

\begin{block}{Перечисление элементов.}
\begin{lstlisting}[language=sh,frame=single]
for planet in Mars Earth Mercury Saturn
 do echo $planet 
done
\end{lstlisting}
  \end{block}
\end{frame}

\begin{frame}[fragile]
\frametitle{ Циклы for. Примеры.}
  \begin{block}{Перечисление цифровой последовательности.}
\begin{lstlisting}[language=sh,frame=single]
for num in 1 2 3 4 5 6 7 8 9 10
 do echo $num
done

for num in $(seq 1 10)  # генерация из внешней команды
 do echo $num
done

for num in {1..10}  # генерация встроенными средствами
 do echo $num
done

\end{lstlisting}
  \end{block}
\end{frame}

\begin{frame}[fragile]
\frametitle{ Циклы for. Примеры.}
  \begin{block}{Перечисление цифровой последовательности C-like }
\begin{lstlisting}[language=sh,frame=single]
for ((i=1;i<11;i++))
do 
  echo $i
done  
\end{lstlisting}
  \end{block}

  \begin{block}{Несколько переменных}
    \begin{lstlisting}[language=sh,frame=single]
for ((a=1, b=1; a <= LIMIT ; a++, b++))
do
  echo -n "$a-$b"
done
    \end{lstlisting}
  \end{block}
\end{frame}

\begin{frame}[fragile]
\frametitle{ Циклы for. Примеры.}
  \begin{block}{Пустые выражения. Результат по умолчанию 1. }
    \begin{lstlisting}[language=sh,frame=single]
for (( i=1; ; i++))
do
  echo $i 
  [ "$i" -eq 10 ] && break 
done
    \end{lstlisting}
  \end{block}
\end{frame}

\begin{frame}[fragile]
\frametitle{ Циклы for. Примеры.}
  \begin{block}{Из аргументов.}
    \begin{lstlisting}[language=sh,frame=single]
for arg
 do echo $arg 
done
    \end{lstlisting}
  \end{block}
  \begin{block}{Из переменной. Запись одной строкой.}
    \begin{lstlisting}[language=sh,frame=single]
for name in $users ; do echo $name ; done
    \end{lstlisting}
  \end{block}
\end{frame}

\begin{frame}[fragile]
\frametitle{Циклы while,until}
\begin{lstlisting}[language=sh,frame=single]
while expr1; ... exprN
do
 op
done
\end{lstlisting}
\end{frame}

\begin{frame}[fragile]
\frametitle{}

\begin{block}{Пример. Перебираем аргументы.}
\begin{lstlisting}[language=sh,frame=single]
while [[ -n $1 ]]
do
    echo $1
    shift
done
\end{lstlisting}
\end{block}

\begin{block}{Пример. Несколько команд.}
\begin{lstlisting}[language=sh,frame=single]
while ((i++))
 read y
do
 echo $i $y
 [[ "$y" = 'stop' ]] && break
done
\end{lstlisting}
\end{block}
\begin{block}{Пример. Бесконечный цикл.}
\begin{lstlisting}[language=sh,frame=single]
while :
do
 x=$RANDOM
 echo $x
 [[ $x -gt 1100 ]] && break
done
\end{lstlisting}
\end{block}
\end{frame}

\begin{frame}[fragile]
\frametitle{ Цикл until. Пример.}
  \begin{block}{Ожидаем хост после перезагрузки.}
    \begin{lstlisting}[language=sh,frame=single]
until ping -q -c 3  $host 1>/dev/null 2>&1 && nc -z $host 22
do 
   sleep 1
   echo unavailable;
done
    \end{lstlisting}
  \end{block}
\end{frame}

\begin{frame}[fragile]
\frametitle{Перенаправление.}
Применяется ко всем командам внутри цикла.
  \begin{block}{Pipe}
    \begin{lstlisting}[language=sh,frame=single]
for name in $users ; do echo $name ; done | wc -l
    \end{lstlisting}
  \end{block}
  \begin{block}{В файл}
    \begin{lstlisting}[language=sh,frame=single]
for name in $users ; do echo $name ; done
(( i=10 )); while (( i > 0 )); do 
    echo "$i"
    (( i-- ))
done > output.txt
    \end{lstlisting}
  \end{block}
\end{frame}

\begin{frame}[fragile]
\frametitle{Внешние команды.}
Массовые операции с файлами.
  \begin{block}{Команда find}
    \begin{lstlisting}[language=sh,frame=single]
find . -name '*.c' -exec stat  {} \;
    \end{lstlisting}
  \end{block}
  \begin{block}{Команда xargs}
    \begin{lstlisting}[language=sh,frame=single]
echo /dev/std* | xargs -n1 readlink
    \end{lstlisting}
  \end{block}
\end{frame}

\begin{frame}[fragile]
    \frametitle{Упражнения}
    \begin{enumerate}
        \item Посчитать сумму кубов чисел от 1 до 100
        \item Вывести в файл 10 случайных чисел от 0 до 80
        \item Построить гистограмму данных из предыдущего файла файла {\bf Hint:} {\tt while read, echo -n }
    \end{enumerate}
\end{frame}
}

\section{Условные операторы}

\mode<all>{\begin{frame}[fragile]
\frametitle{Условные операторы: if}
\begin{itemize}
\item {\tt if;then;else;fi}
\begin{lstlisting}[language=sh,frame=single]
if CONDITIONS
then 
 OPS
[elif] CONDITIONS
[then]
 OPS
[else]
 OPS
fi
\end{lstlisting}
\end{itemize}
\end{frame}

\begin{frame}[fragile]
\frametitle{Условные операторы: case}
\begin{itemize}
\item {\tt case}
\begin{lstlisting}[language=sh,frame=single]
case "$variable" in 
 pattern1) command1
           command2
          ;;
 pattern2|pattern3)
         command3
         command4
        ;;
esac
\end{lstlisting}
\end{itemize}
\end{frame}

\begin{frame}[fragile]
\frametitle{Использование case вместе с getopts}
\begin{lstlisting}[language=sh,frame=single]
while getopts "af:h" Option
do
  case $Option in 
    a) OPTA=1 ;;
    f) OPTFILE=1
       FILENAME=$OPTARG
       ;;
    h) echo "Usage: $0 [-ah] -f <filename>";;
  esac  
done
shift $((OPTIND-1))
\end{lstlisting}
\end{frame}

\begin{frame}[fragile]
\frametitle{Упражнение}
\begin{enumerate}
\item Написать программу, которая по опции {\tt -h } выводит помощь, без опций выводит время в stdout,
с опцией -f выводит время в указаный файл
\end{enumerate}
\end{frame}
}


%\section{Массивы}

%\mode<all>{\begin{frame}[fragile]
\frametitle{Основные операции с массивами}
\begin{itemize}
\item Присвоение значений
 \begin{itemize}
   \item \verb+ a=( "aaa" "bbb ccc" 2 ) +
   \item \verb+ a[1]="aaaa"; a[3]="bb bb";   +
   \item \verb+ a=( [1]="aaa" [10]="bbb"); +
 \end{itemize}
\item Доступ к значениям \verb+ echo ${a[10]} +
\end{itemize}
\end{frame}

\begin{frame}[fragile]
\frametitle{Аттрибуты переменных}
Функция \verb+ declare +
\begin{itemize}
\item {\tt declare -a}
\begin{lstlisting}[language=sh]
declare -a # Список всех массивов
\end{lstlisting} 
\item {\tt declare -i}
\begin{lstlisting}[language=sh]
a="yyy"; echo $a; declare -i a; echo $a;
declare +i a; echo $a
\end{lstlisting}
\item {\tt declare -r}
\item {\tt declare -x}
\item {\tt declare -f}
\end{itemize}
\end{frame}
}

%\section{Функции}

%\mode<all>{\begin{frame}
	\frametitle{Функции}

	\begin{itemize}
		\item Именованные
		\item Неименованные
	\end{itemize}

	Функции в shell могут использоваться как обычные программы, которые:
	\begin{itemize}
		\item Принимают позиционные параметры;
		\item возвращают статус;
		\item Могут использоваться в качестве источника либо приемника 
			при перенаправлениях ввода/вывода.
	\end{itemize}

\end{frame}


\begin{frame}[fragile]
	\frametitle{Функции: синтаксис}
	\begin{itemize}
		\item Классический синаксис: 
			\begin{lstlisting}
function function_name {
command...
} 
			\end{lstlisting}
		\item Портабельный (C-style):
			\begin{lstlisting}
function_name()
{
command...
} 
			\end{lstlisting}

		\item Однострочный:
			\begin{lstlisting}
function_name () { command... ;}
			\end{lstlisting}
  \end{itemize}
\end{frame}

\begin{frame}[fragile]
	\frametitle{Пример (начало)}
	\small
	\begin{lstlisting}
#!/bin/bash

function help {
	echo "Использование: $0 <string>"
	exit 1
}
f1(){
	echo Вызвана функция $0 с $# аргументами
}

f2(){
	while read str; do
		echo $0 прочитана строка: $str
	done
}
	\end{lstlisting}

\end{frame}



\begin{frame}[fragile]
	\frametitle{Пример (окончание)}
	\small
	\begin{lstlisting}
[ $# -eq 0 ] && help

f1 "$@"

{ for ((i=0;i<5;i++));do
	echo $@
done } | f2

exit
	\end{lstlisting}

\end{frame}

\begin{frame}
	\frametitle{}
	\begin{itemize}
		\item 
		\item 
		\item
  \end{itemize}
\end{frame}

\begin{frame}
	\frametitle{}
	\begin{itemize}
		\item 
		\item 
		\item
  \end{itemize}
\end{frame}

\begin{frame}
	\frametitle{}
	\begin{itemize}
		\item 
		\item 
		\item
  \end{itemize}
\end{frame}

\begin{frame}
	\frametitle{}
	\begin{itemize}
		\item 
		\item 
		\item
  \end{itemize}
\end{frame}



}

\end{document}
