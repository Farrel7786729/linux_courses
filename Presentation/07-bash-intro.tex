\documentclass[ignorenonframetext, professionalfonts, hyperref={pdftex, unicode}]{beamer}

\usetheme{Copenhagen}
\usecolortheme{wolverine}

\usepackage[orientation=landscape, size=custom, width=16, height=9.75, scale=0.5]{beamerposter}	

%Packages to be included

\usepackage{textcomp}

\usepackage[russian]{babel}
\usepackage[utf8]{inputenc}
\usepackage[T1]{fontenc}

\usepackage{beamerthemesplit}

\usepackage{ulem}

\usepackage{verbatim}

\usepackage{ucs}
\usepackage{listings}
\lstloadlanguages{C, make, bash}

\lstset{escapechar=`,
	extendedchars=false,
	language=C, 
	tabsize=2, 
	columns=fullflexible, 
%	basicstyle=\scriptsize,
	keywordstyle=\color{blue}, 
	commentstyle=\itshape\color{brown},
%	identifierstyle=\ttfamily, 
	stringstyle=\mdseries\color{green}, 
	showstringspaces=false, 
	numbers=left, 
	numberstyle=\tiny, 
	breaklines=true, 
	inputencoding=utf8x,
	keepspaces=true,
	morekeywords={u\_short, u\_char, u\_long, in\_addr}
	}

\definecolor{darkgreen}{cmyk}{0.7, 0, 1, 0.5}

\lstdefinelanguage{diff}
{
    morekeywords={+, -},
    sensitive=false,
    morecomment=[l]{//},
    morecomment=[s]{/*}{*/},
    morecomment=[l][\color{darkgreen}]{+},
    morecomment=[l][\color{red}]{-},
    morestring=[b]",
}



%%%%%%%%%%%%%%%%%%%%%%%%%%%%%%%%%%%%%%%%%%%%%%%%%
%%%%%%%%%% PDF meta data inserted here %%%%%%%%%%
%%%%%%%%%%%%%%%%%%%%%%%%%%%%%%%%%%%%%%%%%%%%%%%%%
\hypersetup{
	pdftitle={Введение в GNU/Linux},
	pdfauthor={Epam/LLPD}
}





%%%%%% Beamer Theme %%%%%%%%%%%%%

	
\title{Введение в GNU/Linux}
\author{Epam/LLPD}



%%%%%%%%%%%%%%%%%%%%%%%%%%%%%%%%%%%%%%%%%%%%%%%%%
%%%%%%%%%% Begin Document  %%%%%%%%%%%%%%%%%%%%%%
%%%%%%%%%%%%%%%%%%%%%%%%%%%%%%%%%%%%%%%%%%%%%%%%%




\begin{document}

\frame{
	\frametitle{Скрипты на Bash: основы}
	\titlepage
	\vspace{-0.5cm}
	\begin{center}
	%\frontpagelogo
	\end{center}
}

\begin{frame}[fragile]
  \frametitle{Скрипты}
 
  \begin{block}<1->{Shell Script, определение}  

    Последовательность команд Shell.

    Разделитель: перевод строки, ``;''
  \end{block}

  \begin{block}<2->{shebang}
    \verb+#!something+ или чем мы запускаем скрипт. 
    
    По умолчанию : \verb+#!/bin/sh+
  
    Всегда первая строка скрипта.

    Фактически: \verb+/bin/sh scriptname+
  \end{block}

  \begin{block}<3->{Парадоксальные примеры}
    \verb+#!/bin/rm+

    \verb+#!/bin/awk -f+

    \verb+#!/bin/less+
  \end{block}

\end{frame}
\begin{frame}[fragile]
  \frametitle{Переменные shell}
  Какие бывают
  \begin{itemize}
    \item встроенные (в Shell)
      \begin{itemize}
        \item Environment (наследуются дочерними процессами)
      \item Внутренние (PS1,PS2,...)
      \end{itemize}
    \item пользовательские 
  \end{itemize} 
\end{frame}
\begin{frame}[fragile]
  \frametitle{Работа с переменными}
  \begin{itemize}
    \item \alert{set} Список всех переменных (use grep!)
    \item \alert{\$VAR} взять значение переменной VAR
    \item \alert{''\${VAR}blah-blah''} подстановка переменной внутрь строки
    \item \alert{unset} - сброс значения 
    \item \alert{VAR=<выражение>} (без пробелов!)
   \end{itemize}
 \end{frame}
\begin{frame}[fragile]
  \frametitle{Дефолтные переменные скрипта}
  \begin{itemize}
    \item \alert{\$1,\$2,...,\$9,\${10},\${11}...}
    \item 
  \end{itemize}
\end{frame}


\end{document}
