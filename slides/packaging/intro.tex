\begin{frame}
	\frametitle{Про "DLL hell"}
	
	\begin{block}{Устанавливаем программу}
	А что же с библиотеками?
	\end{block}

	\pause

	\begin{columns}
		\column{0.5\textwidth}
		\begin{block}{"В системе все есть!"}
		\begin{itemize}
			\item Oh, really???
			\item И нужной версии?
			\item А API и ABI точно не менялись?
			\item А если библиотек несколько версий?
			\item А если нужны дополнительные программы?
		\end{itemize}
		\end{block}
		\pause
		\column{0.5\textwidth}
		\begin{block}{"Всё своё, ношу с собой!"}
		\begin{itemize}
			\item А как насчет объема?
			\item Использование памяти.
			\item А что насчет лицензий?
			\item И все-таки порядок загрузки...
			\item Не спасает от проблем с 3rd-party ПО.
		\end{itemize}
		\end{block}
	\end{columns}
\end{frame}

\begin{frame}
	\frametitle{Хаос}

	\begin{center}
		"Даешь каждой платформе и языку собственную систему управления пакетами!"
	\end{center}

	\begin{block}{Увы, мы не в идеальном мире}
		\begin{itemize}
			\item Дистрибутивы: rpm\{4,5\}, deb, portage, pacman... и куча модификаций...
			\item Дополнительный софт: {\tt ./configure; make; make install}
			\item Java: {\tt ivy, ant, maven, gradle}
			\item Ruby: gem
			\item Perl: CPAN
			\item Python: pip + PyPi
		\end{itemize}
	\end{block}

\end{frame}

\begin{frame}
	\frametitle{Разработка и использование в реальной системе}
	
	\begin{block}{Build-time vs Run-time}

		\begin{enumerate}
			\item Не все, что нужно во время компиляции, должно быть установлено в конечной системе.
			\item Не все, что нужно для работы программы, необходимо устанавливать на сборочной системе.
		\end{enumerate}
	\end{block}
%TODO перенести после описания структуры каталогов
	\begin{block}{Чистое сборочное окружение}
		\begin{itemize}
			\item Воспроизводимость сборки 
			\item Контроль зависимостей
			\item Контроль автоматически "подхваченных" зависимостей
		\end{itemize}
	\end{block}
\end{frame}

