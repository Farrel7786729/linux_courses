\begin{frame}{Практика: создание тестовой среды}
	\begin{itemize}
		\item Подготовка диска
			\begin{itemize}
				\item Создать пустой файл размером в 1GB и отобразить на устройство
					/dev/loop ({\tt dd, losetup})
				\item Создать группу томов на базе этого устройства ({\tt pvcreate, vgcreate})
				\item Выделить 75\% объема под логический диск ({\tt lvcreate})
				\item Скопировать содержимое виртуального диска в полученный логический том ({\tt dd})
				\item Создать снимок логического тома на 20\% размера ({\tt lvcreate})
			\end{itemize}
	\end{itemize}
\end{frame}

\begin{frame}{Практика: создание тестовой среды}
	\begin{itemize}
		\item Запуск виртуальных машин
			\begin{itemize}
				\item {\tt modprobe kvm-intel} 
				\item {\tt modprobe tun} 
				\item {\tt kvm -m 512 -boot c -net nic -net tap,ifname=tap0,script=no,downscript=no /dev/VG0/origin } 
				\item {\tt kvm -m 512 -boot c -net nic -net tap,ifname=tap1,script=no,downscript=no /dev/VG0/snap } 
			\end{itemize}
			\pause
		\item Настройка сети на хосте
			\begin{itemize}
				\item Создать мост {\tt brctl} и назначить ему адрес {\tt ifconfig/ip}
				\item Поднять виртуальные интерфейсы {\tt ifconfig/ip}
				\item Добавить дополнительный интерфейс и виртуальные интерфейсы к мосту {\tt brctl}
			\end{itemize}
			\pause
		\item Настройка сети на виртуалках
			\begin{itemize}
				\item Назначить адрес устройству eth0 {\tt ifconfig/ip}
				\item Проверить {\tt ping}
			\end{itemize}

	\end{itemize}
\end{frame}


