
\begin{frame}{''soft-emulated'' интерфейсы}
	\begin{block}{TUN/TAP}

        Интерфейсы, эмулирумые программно, позволяющие процессам ''работать'' с интерфейсом в userspace.
		Возможно понадобится сделать: {\tt modprobe tun}

	\end{block}

	\begin{block}{old-style}
		\begin{itemize}
			\item Добавить интерфейс TUN -- {\tt tunctl -n -t <ifacename>}
			\item Добавить интерфейс TAP -- {\tt tunctl -p -t <ifacename>}
			\item Удалить интерфейс -- {\tt tunctl -d <ifacename>}
		\end{itemize}
	\end{block}
	\begin{block}{iproute2}
		\begin{itemize}
			\item Добавить интерфейс tun/tap (<mode>) -- \\ {\tt ip tuntap add <ifacename> mode <mode>}
			\item Удалить интерфейс -- \\ {\tt ip tuntap del <ifacename> mode <mode>}
		\end{itemize}
	\end{block}

\end{frame}

%TODO
	\begin{block}{Практика}
		\begin{itemize}
			\item Создать интерфейс TAP с именем {\tt mytap}
			\item Создать мост с именем {\tt mybr}
			\item Назначить интерфейсу {\tt mytap} адрес {\tt 192.168.0.<n>/24}
			\item Добавить интерфейсы {\tt mytap} и {\tt eth0} к мосту {\tt mybr}
			\item Запустить {\tt tcpdump} на интерфейсах {\tt mybr} и {\tt mytap}
			\item Запустить {\tt ping} соседа
		\end{itemize}
	\end{block}



