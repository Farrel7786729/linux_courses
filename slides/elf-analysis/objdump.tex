%%
%% 

\begin{frame}
	\frametitle{objdump}

	\begin{block}{objdump}
		Утилита для просмотра информации по object файлам.
	\end{block}

	\pause

	\begin{block}{Упражнение}
		\begin{itemize}
			\item {\tt objdump -{}-file-headers main-A\_B} -- получить заголовок файла
			\item {\tt objdump -{}-section-headers main-A\_B} -- перечисление секций (для линковки)
			\item {\tt objdump -{}-syms main-A\_B} -- список таблицы символов
			\item {\tt objdump -{}-dynamic-syms main-A\_B} -- список динамических символов (как nm -u)
		\end{itemize}

	\end{block}
\end{frame}


\begin{frame}
	\frametitle{Дизассемблер objdump}

	\begin{block}{Упражнение: внутренниий мир программы}
		\begin{itemize}
			\item {\tt objdump -{}-disassemble main-A\_B} -- получить дизассемблированный файл
			\item {\tt objdump -{}-disassemble -j.text main-A\_B} -- 
				получить дизассемблированную исполняемую секцию {\tt .text}
			\item {\tt objdump -{}-disassemble -j.text main-A\_B} -- 
				получить дизассемблированную исполняемую секцию {\tt .text} 
				и узнать где в файле располагаются инструкции
		\end{itemize}
	\end{block}
\end{frame}

\begin{frame}
	\frametitle{Что же наделал компилятор?}

	\begin{block}{Упражнение: анализируем вместе с исходником}
		\begin{itemize}
			\item Добавить в исходник {\tt main.c} объявление переменной и пустой цикл
			\item Добавить в {\tt Makefile} опции {\tt -g -O0}
			\item {\tt objdump -{}-source -{}-line-numbers main-A\_B}
			\item {\tt objdump -{}-source -{}-line-numbers -{}-file-offsets main.o}
			\item Изменить в {\tt Makefile} опцию {\tt -O0} на {\tt -O3}
			\item {\tt objdump -{}-source -{}-line-numbers -{}-file-offsets main.o}
			\item Сделать выводы ;-)
		\end{itemize}
	\end{block}
\end{frame}

