\begin{frame}
 \frametitle{Статическая и динамическая линковка}
 \begin{itemize}
   \item Статическая линковка
     \begin{columns}
       \column{0.4\textwidth}
       \begin{center}
         Достоинства 
       \end{center}
       \begin{itemize}
         \item Быстрая загрузка
         \item Большая переносимость
       \end{itemize}
       \column{0.4\textwidth}
       \begin{center}
         Недостатки
       \end{center}
       \begin{itemize}
         \item Много места на диске
         \item Много места в памяти
         \item Перелинковка при изменениях
       \end{itemize}
     \end{columns}
   \item Динамическая линковка
     \begin{columns}
       \column{0.4\textwidth}
       \begin{center}
         Достоинства 
       \end{center}
       \begin{itemize}
         \item Экономия памяти и диска
         \item Не нужна перелинковка
       \end{itemize}
       \column{0.4\textwidth}
       \begin{center}
         Недостатки
       \end{center}
       \begin{itemize}
         \item Медленная загрузка
         \item Структура директорий должна совпадать
       \end{itemize}
        
     \end{columns}
    \item{Линковка во время выполнения}
 \end{itemize}
\end{frame}
\begin{frame}[fragile]
 \frametitle{Создание статических библиотек}
\begin{lstlisting}[language=sh]
gcc -g -Wall -c  file1.c
gcc -g -Wall -c  file2.c
ar rcs libmylib.a file1.o file2.o
#ranlib libmylib.a
gcc -o file3 file3.c -L. -lmylib # Использование
\end{lstlisting}
\end{frame}

\begin{frame}[fragile]
  \frametitle{Создание динамических разделяемых библиотек}
\begin{lstlisting}[language=sh]
gcc -g -fpic -c -Wall   file1.c; gcc -g -Wall -fpic -c  file2.c
gcc -g -shared -Wl,-soname,libmylib.so.0 -o libmylib.so.0.0 
cp libmylib.so.0.0 /usr/local/lib/
ldconfig 
gcc -g -o file3 file3.o -lmylib
\end{lstlisting}
\end{frame}

\begin{frame}
 \frametitle{Проблемы при линковке}
 \begin{itemize}
   \item Underlinking
   \item Зависимость от порядка при опции {\tt --as-needed} 
   \item Overlinking
   \item Несовместимость версий (soname etc.)
 \end{itemize}
\end{frame}

\begin{frame}[fragile]
  \frametitle{Поиск динамических библиотек}
  \begin{itemize}
    \item {\tt -rpath} Опция линкера
    \item \verb+ LD_LIBRARY_PATH +
    \item {\tt ldconfig}
    \item {\tt /etc/ld.so.conf}
    \item {\tt /etc/ld.so.cache}
  \end{itemize}
\end{frame}

\begin{frame}
  \frametitle{Упражнение}
  \begin{itemize}
    \item Оформить функцию {\tt hello(char * user) } которая приветствует пользователя в отдельный файл
    \item Скомпилировать этот файл и создать из него разделяемую динамическую библиотеку {\tt (libhello.so)}
    \item Создать программу вызывающую функцию {\tt hello()} и слинковать ее с динамической библиотекой
  \end{itemize}
\end{frame}

\begin{frame}
	\frametitle{Пример: последовательность линковки библиотек}

	\begin{enumerate}
		\item Создать 2 библиотеки ({\tt liba.c} и {\tt libb.c})
			с одинаковой функцией {\tt void test(void)}, но разным функционалом\\
			(например вывести разные строки на экран)
		\item Создать программу ({\tt main.c}), которая будет использовать 
			библиотечную функцию {\tt test()}
		\item Линковать с библиотеками в различной последовательности и 
			посмотреть результат выполнения команды
			\begin{itemize}
				\item {\tt -lb -la -L.}
				\item {\tt -la -lb -L.}
			\end{itemize}
	\end{enumerate}
\end{frame}
