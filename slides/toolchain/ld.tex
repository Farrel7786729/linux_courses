\begin{frame}[fragile]
	\frametitle{Бинарная совместимость}

	\begin{block}{Application binary interface (ABI)}
		\begin{itemize}
				\item размер и выравнивание данных
				\item расположение структур
				\item соглашения по вызовам
				\item соглашения по использованию регистров
				\item форматы объектных файлов
		\end{itemize}
	\end{block}
\end{frame}



\begin{frame}
 \frametitle{Статическая и динамическая линковка}
 \begin{itemize}
   \item Статическая линковка
     \begin{columns}
       \column{0.4\textwidth}
       \begin{center}
         Достоинства 
       \end{center}
       \begin{itemize}
         \item Быстрая загрузка
         \item Большая переносимость
       \end{itemize}
       \column{0.4\textwidth}
       \begin{center}
         Недостатки
       \end{center}
       \begin{itemize}
         \item Много места на диске
         \item Много места в памяти
         \item Перелинковка при изменениях
       \end{itemize}
     \end{columns}
   \item Динамическая линковка
     \begin{columns}
       \column{0.4\textwidth}
       \begin{center}
         Достоинства 
       \end{center}
       \begin{itemize}
         \item Экономия памяти и диска
         \item Не нужна перелинковка
       \end{itemize}
       \column{0.4\textwidth}
       \begin{center}
         Недостатки
       \end{center}
       \begin{itemize}
         \item Медленная загрузка
         \item Структура директорий должна совпадать
       \end{itemize}
        
     \end{columns}
    \item{Линковка во время выполнения}
 \end{itemize}
\end{frame}

\begin{frame}[fragile]
 \frametitle{Создание статических библиотек}
\begin{lstlisting}[language=sh]
gcc -g -Wall -c  file1.c
gcc -g -Wall -c  file2.c
ar rcs libmylib.a file1.o file2.o
#ranlib libmylib.a = ar s libmylib.a
gcc -o file3 file3.c -L. -lmylib # Использование
\end{lstlisting}

\end{frame}

\begin{frame}
	\frametitle{Упражнение: статическая библиотека}

	\begin{enumerate}
		\item Создать "библиотеку" ({\tt liba.c})
			с функцией {\tt void test(void)}, выводящей строку {\tt ``LibA''} на экран
		\item Создать статическую библиотеку {\tt liba.a} с помощью {\tt ar}
		\item Создать программу ({\tt main.c}), которая будет использовать 
			библиотечную функцию {\tt test()}
		\item Слинковать {\tt main.o} со статической библиотекой:
			\begin{itemize}
				\item {\tt gcc -o main -L. -la main.o }
				\item {\tt gcc -o main main.o -L. -la }
			\end{itemize}
		\item Создать статически исполняемый файл: 
			\begin{itemize}
				\item {\tt gcc -o main main.o -L. -la -static}
			\end{itemize}

	\end{enumerate}
\end{frame}


\begin{frame}
	\frametitle{Разделяемые библиотеки}

	\begin{block}{Именование}
	
		{\bf Soname}:
		\begin{itemize}
				\item Префикс {\tt lib}
				\item Имя
				\item Суффикс {\tt .so}
				\item Версия
		\end{itemize}

		{\bf Real name}: soname, но версия включает в себя минорные изменения.

		{\bf Link name}: soname, но без версии

	\end{block}

	\begin{block}{Упражнение}
		Проанализировать файлы, находящиеся в {\tt /usr/lib}\footnote{Зависит от архитектуры}
	\end{block}


\end{frame}

\begin{frame}[fragile]
  \frametitle{Поиск динамических библиотек}
  \begin{block}{Динамический загрузчик}
     \begin{itemize}
	\item {\tt ld.so} -- для формата {\tt a.out}
	\item {\tt ld-linux.so} -- для формата {\tt ELF}
    \end{itemize}
  \end{block}
  \begin{itemize}
	\item {\tt ldconfig}
	\item {\tt /etc/ld.so.conf}
	\item {\tt /etc/ld.so.cache}
	\item {\tt /etc/ld.so.preload}
	\item {\it Опция линкера} {\tt -rpath}
  \end{itemize}
\end{frame}


\begin{frame}[fragile]
  \frametitle{Создание динамических разделяемых библиотек}
\begin{lstlisting}[language=sh]
gcc -g -fPIC -c -Wall   file1.c; gcc -g -Wall -fPIC -c  file2.c
gcc -g -shared -Wl,-soname,libmylib.so.0 -o libmylib.so.0.0 
cp libmylib.so.0.0 /usr/local/lib/
ldconfig 
gcc -g -o file3 file3.o -lmylib
\end{lstlisting}

	\begin{block}{Флаги}
	
		{\bf -fpic} или {\bf -fPIC} -- создание позиционно-независимого кода в объектных файлах.

		{\bf -shared} -- создание разделяемой библиотеки

		{\bf -Wl,-soname,libmylib.so.0} -- присвоение ``soname'' имени библиотеке

	\end{block}

\end{frame}

\begin{frame}
 \frametitle{Проблемы при линковке}
 \begin{itemize}
   \item Underlinking
   \item Зависимость от порядка при опции {\tt -{}-as-needed} 
   \item Overlinking
   \item Несовместимость версий (soname etc.)
 \end{itemize}
\end{frame}

\begin{frame}[fragile]
	\frametitle{GCC environment variables}

	\begin{block}{Build time}
		\begin{itemize}
			\item {\tt CC} -- (пере)определение компилятора
			\item {\tt CPATH, C\_INCLUDE\_PATH} -- расположение включаемых файлов для {\tt -I}
			\item {\tt LIBRARY\_PATH} -- расположение динамически линкуемых библиотек {\tt -l} (link time)
		\end{itemize}
	\end{block}

	\begin{block}{Runtime}
		\begin{itemize}
			\item {\tt LD\_LIBRARY\_PATH} -- расположение динамически линкуемых библиотек (runtime)
			\item {\tt /lib64/ld-linux-x86-64.so.2 -{}-library-path PATH} -- 
			      расположение динамически линкуемых библиотек (runtime)
			\item {\tt LD\_PRELOAD} -- пути к библиотекам, загружаемым до программы
		\end{itemize}
	\end{block}
%			\item {\tt CFLAGS} -- флаги для компилтора
%			\item {\tt LDFLAGS} -- флаги для линкера

\end{frame}



