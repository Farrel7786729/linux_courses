\begin{frame}
 \frametitle{Системы автоматического документирования}
 \begin{itemize}
  \item Donald Knuth, literate programming, 1983
  \item perldoc, etc. 1994
  \item javadoc, 1995
  \item Doxygen, 1997
 \end{itemize}
\end{frame}

\begin{frame}
 \frametitle{Как использовать}
 \begin{itemize}
  \item Добавить комментарии в формате Doxygen в исходники
  \item \texttt{doxygen -g Doxyconfig}
  \item Редактировать Doxyconfig
  \item doxygen Doxyconfig
 \end{itemize}
\end{frame}

\begin{frame}[fragile]
 \frametitle{Образец комментариев для Doxygen}
\begin{lstlisting}[language=C]
/**
  \brief Краткое описание функции

  Детальное описание
  @param parameter1 Описание первого параметра
  @see Ссылка на другую функцию

*/
int function(int parameter1) 
\end{lstlisting}
\end{frame}

\begin{frame}[fragile]
 \frametitle{Некоторые важные параметры в конфигурационном файле}
 \begin{itemize}
  \item \verb+PROJECT_NAME+
  \item \verb+PROJECT_NUMBER+
  \item \verb+INPUT+  
  \item \verb+RECURSIVE+
  \item \verb+EXCLUDE+
  \item \verb+EXCLUDE_PATTERNS+
  \item \verb+EXTRACT_ALL+
 \end{itemize}
\end{frame}

\begin{frame}
 \frametitle{Упражнение}
  \begin{itemize}
    \item В проекте helloworld откомментировать каждую функцию в стиле doxygen
    \item Сгенерировать документацию в формате pdf
  \end{itemize}
\end{frame} 
