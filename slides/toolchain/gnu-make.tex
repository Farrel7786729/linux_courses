\begin{frame}[fragile]
	\frametitle{Библиотека}

	\begin{columns}
		\column{0.5\textwidth}
		{\tt name.h}:

		\begin{lstlisting}
extern char* get_name();
		\end{lstlisting}
		\column{0.5\textwidth}

		{\tt name.c}:

		\begin{lstlisting}
#include "name.h"

char* get_name()
{
    return LIB_NAME;
}
		\end{lstlisting}
	\end{columns}
	
	\begin{center}
	Чего не хватает в хидере? А ещё?
\end{center}
\pause
	\begin{lstlisting}
gcc --shared -o libname.so name.c -DLIB_NAME=\"libname.so\"
	\end{lstlisting}
\end{frame}

\begin{frame}[fragile]
	\frametitle{Программа}

	{\tt hello.с:}

	\begin{lstlisting}
#include <stdio.h>
#include "name.h"

int main(int argc, char** argv)
{
    printf("Hello, %s\n", get_name());
    return 0;
}
	\end{lstlisting}

	Сборка:

	\begin{verbatim}
gcc hello.c -c -o hello.o
gcc hello.o -L=. -lname -o hello
	\end{verbatim}
\end{frame}

\begin{frame}
  \frametitle{Задачи Make}
  \begin{itemize}
     \item Набор правил для сборки сложных проектов одной командой
     \item Поддержание зависимостей
       \begin{itemize}
         \item Пересборка только изменившихся частей
         \item Определение необходимости пересборки
       \end{itemize}
  \end{itemize}
\end{frame}

\begin{frame}[fragile]
	\frametitle{Правила make}

	Общий синтаксис:
	\begin{verbatim}
<target>: <dependency0> <dependency1>
<TAB><command0>
<TAB><command1>
<TAB><command2>
	\end{verbatim}

	Абстрактные цели:
	\begin{verbatim}
.PHONY <target0> ... <targetN>
	\end{verbatim}
\end{frame}

\begin{frame}[fragile]
	\frametitle{Makefile}

	\begin{lstlisting}[language=make]
.PHONY release
release: hello
hello: hello.o libname.so
    gcc hello.o -L=. -lname -o hello

hello.o: hello.c name.h
    gcc hello.c -c -Wall -fPIC -o hello.o

libname.so: name.c
    gcc --shared -o libname.so name.c -DLIB_NAME=\"libname.so\"

name.c: name.h

hello.c: name.h
	\end{lstlisting}
\end{frame}



\begin{frame}[fragile]
	\frametitle{Переменные в Makefile}
	\begin{lstlisting}[language=make]
all_flags = $(compiler_flags) $(linker_flags)
no_flags := $(compiler_flags) $(linker_flags)
compiler_flags = -Wall -O3
linker_flags := -fPIC
compiler_flags += -pipe

show-flags:
    @echo "Compiler flags" $(compiler_flags)
    @echo "Linker flags" $(linker_flags)
    @echo "All flags" $(all_flags)
    @echo "No flags" $(no_flags)
	\end{lstlisting}
\end{frame}
\begin{frame}[fragile]
	\frametitle{Псевдонимы и шаблонные правила}

	\begin{itemize}
		\item {\tt \$@} -- Имя цели обрабатываемого правила
		\item {\tt \$<} -- Имя первой зависимости обрабатываемого правила 
		\item {\tt \$\^{}} -- Список всех зависимостей обрабатываемого правила
	\end{itemize}

	\begin{lstlisting}[language=make]
%.o: %.c
    gcc $^ -o $@

VPATH := . somedir
	\end{lstlisting}
\end{frame}

\begin{frame}[fragile]
\frametitle{Скрытые (implicit) правила}
\begin{block}{Упражнение}
В пустой директории создайте файл hello.c, который выводит hello, world
\begin{lstlisting}[language=bash]
make hello
\end{lstlisting}
\end{block}
\pause
\begin{center}
Некоторые скрытые правила
\end{center}
\begin{lstlisting}[language=make]
%.o:%.c
        $(CC) -c $(CPPFLAGS) $(CFLAGS) $^
%.o:%.cc
        $(CXX) -c $(CPPFLAGS) $(CXXFLAGS) $^
%:%.o
        $(CC) -o $@ $(LDFLAGS) $^ $(LOADLIBES) 
\end{lstlisting}
\end{frame}
\begin{frame}[fragile]
	\frametitle{Функции в make}

	\begin{itemize}
		\item wildcard -- список всех файлов по паттерну,  находящихся в директории проекта
		\item addprefix -- подстановка первого аргумента к каждому слову из второго аргумента
		\item patsubst -- замена в списке
		\item include -- как и {\tt \#include} в {\tt c/c++}
	\end{itemize}

	\begin{lstlisting}[language=make]
src_dirs := core stuff generic
src_dirs := $(addprefix ../../,  $(src_dirs))

$(patsubst %.c, %.o, $(wildcard *.c))
	\end{lstlisting}

\end{frame}

\begin{frame}[fragile]
	\frametitle{Автоматические зависимости}

	Создание зависимостей средствами gcc:

	\begin{verbatim}
gcc -M -MM -MD -MMD
	\end{verbatim}

	\begin{lstlisting}[language=make]
include $(wildcard *.d)
	\end{lstlisting}
\end{frame}
\begin{frame}[fragile]
	\frametitle{Практическое задание}

	\begin{enumerate}
		\item добавьте цель {\tt clean}
		\item перепишите с использованием шаблонных правил
		\item заведите переменные {\tt CFLAGS}, {\tt LDFLAGS}
		\item сохраняйте промежуточные файлы в каталог {\tt out/}
		\item создайте правило {\tt debug:} для сборки в отладочном режиме
	\end{enumerate}
\end{frame}

