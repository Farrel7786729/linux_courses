\begin{frame}{init}
	Менеджер управления работой системой и сервисами.
	
	\bigskip

	\center{\large PID = 1}

	\bigskip

	\begin{block}{Наиболее известные}
		\begin{itemize}
			\item SysVInit
			\item systemd
			\item upstart
		\end{itemize}
	\end{block}
\end{frame}

\begin{frame}{SysVInit}
	\begin{block}{Управление}
		\begin{itemize}
			\item kernel boot parameters: <N> -- runlevel
			\item утилита {\tt runlevel}
			\item утилита {\tt init}
		\end{itemize}
	\end{block}

	\scriptsize
	\begin{block}{Runlevel}
		\begin{table}
			\begin{tabular}{| c | l | }
			\hline
			Runlevel & Описание\\
			\hline
			0	& Выключить систему \\
			1,s,single & Однопользовательский режим \\
			2	& Многопользовательский режим без графики. Без сетевых сервисов.\\
			3	& Многопользовательский режим без графики. Полноценная сеть. \\
			4	& Определяется на хосте\\
			5	& Многопользовательский режим с графикой.\\
			6	& Перезагрузка\\
			emergency & Аварийная оболочка \\
			\hline
			\end{tabular}
		\end{table}
	\end{block}
\end{frame}

\begin{frame}{SysVInit: сервисы}
	\begin{block}{Управление}
		\begin{itemize}
			\item утилита {\tt service}
			\item утилита {\tt chkconfig}
		\end{itemize}
	\end{block}

	\begin{block}{Сервисы}
		\begin{itemize}
			\item {\tt /etc/rc.d/init.d}
			\item {\tt /etc/rc.d/rc.N}\footnote{N=runlevel}
		\end{itemize}
	\end{block}
\end{frame}

\begin{frame}{systemd}
	\begin{block}{Управление}
		\begin{itemize}
			\item kernel boot parameters\\
				{\tt systemd.unit=rescue.target} \\
			\item утилита {\tt systemctl} \\
				{\tt systemctl isolate multi-user.target} \\
				{\tt systemctl set-default single.target}
		\end{itemize}
	\end{block}

	\begin{block}{targets}
		\tiny
		\begin{table}
			\begin{tabular}{| c | l | l | }
			\hline
			Runlevel & Описание\\
			\hline
			0	& poweroff.target & Выключить систему \\
			1,s,single & rescue.target  & Однопользовательский режим \\
			2	& multi-user.target & Многопользовательский режим без графики. Без сетевых сервисов.\\
			3	& multi-user.target & Многопользовательский режим без графики. Полноценная сеть. \\
			4	& multi-user.target & Определяется на хосте\\
			5	& graphical.target & Многопользовательский режим с графикой.\\
			6	& reboot.target & Перезагрузка\\
			emergency & emergency.target & Аварийная оболочка \\
			\hline
			\end{tabular}
		\end{table}
	\end{block}
\end{frame}

\begin{frame}{systemd: сервисы}
	\begin{block}{Управление}
		\begin{itemize}
			\item утилита {\tt systemctl}
		\end{itemize}
	\end{block}

	\begin{block}{Сервисы}
		\begin{itemize}
			\item {\tt /lib/systemd/system/}
			\item {\tt /etc/systemd/system/}
		\end{itemize}
	\end{block}
\end{frame}
