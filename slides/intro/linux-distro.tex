\begin{frame}{Дистрибутив ОС GNU/Linux}
	\begin{block}{ Определение}
		\only<1>{\center{\bf{?}}}
		\pause
		\only<2->{Набор программного обеспечения на базе ядра Linux, распространяющийся как единое целое.}
	\end{block}
\end{frame}


\begin{frame}{Задачи дистрибутива}
	\begin{itemize}
		\item Предоставление комплекта ПО (ядро + утилиты)
		\item Средства установки и настройки
		\item Средства обновления
	\end{itemize}
\end{frame}

\begin{frame}{Различия между дистрибутивами}

	\only<1>{\Large\center{\bf{?}}}
	\pause
	\only<2->{\Large\center{\bf{Цели!!!}}}

	\bigskip
	\normalsize

	\pause

	\begin{itemize}
		\begin{columns}
		\column{0.4\textwidth}
			\item Инсталлятор
			\item Первичные настройки
			\item Средства управления
			\item Набор ПО
		\column{0.4\textwidth}
			\item Менеджер пакетов
			\item Формат распространения ПО
			\item Пути к файлам
			\item Система сборки ПО
		\end{columns}
	\end{itemize}
\end{frame}

\begin{frame}{Дистрибутивы}
	\begin{itemize}
		\begin{columns}
		\column{0.3\textwidth}
			\item RedHat
			\item Fedora Core
			\item CentOS
			\item Scientific Linux
			\item Oracle Unbreakable Linux
		\column{0.3\textwidth}
			\item Slackware 
			\item Gentoo
			\item Arch
			\item OpenSUSE
			\item ALT Linux 
		\column{0.3\textwidth}
			\item Debian
			\item Ubuntu
			\item Mint
			\item Knoppix
			\item BackTrack
		\end{columns}
	\end{itemize}
\end{frame}
