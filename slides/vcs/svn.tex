\begin{frame}
 \frametitle{Базовые команды для работы с svn клиентом}
  \begin{itemize}
   \item svn help
   \item svn checkout 
   \item svn import
   \item svn update
   \item svn commit
   \item svn status
   \item svn copy
   \item svn switch
  \end{itemize}
\end{frame}

\begin{frame}
 \frametitle{Типичная последовательность действий при работе с svn}
 \begin{itemize}
  \item svn import my_project svn://myhost.com/my_project/trunk
  \item $\dots$
  \item svn checkout svn://myhost.com/my_project/trunk myproject
  \item Редактируем файлы в директории myproject
  \item svn status -- смотрим изменения
  \item svn diff
  \item svn update 
  \item Разрешаем конфликты
  \item svn commit --user me --password mypass -m "Some changes"
  \item svn update
 \end{itemize}
\end{frame}

\begin{frame}
 \frametitle{Типичная структура svn репозитория}
 \begin{itemize}
  \item project1
    \begin{itemize}
      \item trunk
        \begin{itemize}
          \item file1.txt
          \item $\dots$
        \end{itemize}
      \item tags
      \item branches
         \begin{itemize}
            \item branch1
            \item branch2
         \end{itemize}
     \end{itemize}
  \item project2

 \end{itemize}
\end{frame}

\begin{frame}
 \frametitle{Как создавать svn репозиторий}
 \begin{itemize}
   \item svnadmin
     \begin{itemize}
       \item svnadmin help
       \item svnadmin create <путь к директории>
     \end{itemize}
    \item svnserve
      \begin{itemize}
        \item svnserve -d -r <путь к репозиторию>
        \item Файлы настройки: conf/passwd, conf/svnserve.conf
      \end{itemize}
\end{frame}

\begin{frame}
 \frametitle{Упражнение}
  \begin{itemize}
    \item Поднять svn репозиторий на localhost
    \item Внести в него файлы из проекта rpm
    \item Сделать svn checkout всего проекта в отдельную директорию
    \item Отредактировать несколько файлов в созданной директории
    \item Закоммитить изменения в центральный репозиторий
    \item Создать ветку от исходной ревизии
  \end{itemize}
\end{frame}
