\begin{frame}[fragile]{Определение(не совсем формальное)}
	\textbf{Shell} -- приложение, обеспечивающее выполнение других приложений и их взаимодействие, а также представляющая услуги командной строки. 
	\begin{center}
	 или
	\end{center}
	\textbf{Shell} -- приложение, обеспечивающее доступ к основным функциям ядра.

	\pause
	\vspace{0.5in}
	Пример shell из Windows-world -- cmd.exe
	\vspace{0.5in}

	Минимальный дистрибутив Linux -- ядро + shell 

\end{frame}

\begin{frame}[fragile]{Основные типы shell в Unix}
  \begin{itemize}
    \item Bourne shell совместимые
      \begin{itemize}
        \item \textbf{sh} исходная bourne shell (Steve Bourne, 1978)
        \item \textbf{ksh} Korn shell (David Korn, 1983)
        \item \textbf{ash} $[$BSD$]$ Almquist shell (Kenneth Almquist,1989)  
        \item \textbf{bash} $[$GPL$]$ Bourne-again shell (Brian Fox, 1989)
        \item \textbf{zsh} $[$BSD$]$ Z shell (Paul Falstad,1990)
        \item \textbf{/bin/sh} Указывает на POSIX-совместимую shell
      \end{itemize}
  \item C shell совместимые
      \begin{itemize}
        \item \textbf{csh}  Исходная С shell (Bill Joy, 1978)
        \item \textbf{tcsh} $[$BSD$]$ TENEX C shell (Ken Greer, 1981)
       \end{itemize}
  \end{itemize}
\end{frame}

\begin{frame}[fragile]{Маленькое упражнение}
\begin{lstlisting}[language=bash]
cat /etc/shells
ls -l <filename> # для каждого элемента /etc/shells
readlink -e <filename> 
\end{lstlisting}
\end{frame}


