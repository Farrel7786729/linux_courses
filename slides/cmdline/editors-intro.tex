\begin{frame}{Редакторы}
	\begin{itemize}
		\item Интерактивные
			\begin{itemize}
				\item vi
					\begin{itemize}
						\item Есть почти везде
					\end{itemize}
				\item vim
				\item emacs
			\end{itemize}
		\item Поточные
			\begin{itemize}
				\item {\tt ed}
				\item {\tt sed}
				\item {\tt awk}
			\end{itemize}
	\end{itemize}
\end{frame}

\begin{frame}[fragile]{Метасимволы - напоминание}
	\begin{block}{grep, sed, awk}
	\end{block}
	\begin{itemize}
		\item {\tt .} -- любой символ за исключением пустой строки
		\item {\tt *} -- любоe количество символов, которые стоят перед {\tt *}
		\item {\tt \^{}} -- начало строки
		\item {\tt \$} -- конец строки
		\item {\tt [...]} -- любой символ из заключенных в скобки
	\end{itemize}
\end{frame}

\begin{frame}[fragile]{sed}
	\begin{block}{Сценарии}
		{\tt [ addr [ ,  addr ] ] cmd [ args ]}
	\end{block}

	\tiny
	\begin{block}{Команды}
		\begin{itemize}
		  \item {\tt a, i} -- добавить строку после (перед) текущей
			  \begin{verbatim} who | sed -e 'a Text' \end{verbatim}
		  \item {\tt c} -- удалить строку и заменить на текст
			  \begin{verbatim} who | sed -e "/$USER/ c Юзверь" \end{verbatim}
		  \item {\tt d} -- удалить строку
			  \begin{verbatim} who | sed -e '2,4 d' \end{verbatim}
			  \begin{verbatim} who | sed -e '/pts/ d' \end{verbatim}
		  \item {\tt s} -- замена по регулярному выражению
			  \begin{verbatim} who | sed -e "s/$USER/Юзверь/g" \end{verbatim}
		\end{itemize}
	\end{block}
	\pause
	\begin{block}{Задача}
		С помощью {\tt find} найти все вложенные директории в {\tt /etc} и 
		''переделать'' их в windows-style
	\end{block}
\end{frame}


