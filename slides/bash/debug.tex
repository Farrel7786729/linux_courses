\begin{frame}[fragile,allowframebreaks]
	\frametitle{Отладка скриптов с помощью set}
	
	Используется команда {\tt set} для активизации различных режимов работы {\tt bash}.

	Включение режима производится с помощью ''-'',\\
	а отключение -- ''+''.

	\begin{itemize}
		\item {\tt set -v} -- вывод на экран исполняемой строки
			\begin{lstlisting}[language=sh]
set -v
echo $HOME
			\end{lstlisting}

		\item {\tt set -x} -- вывод на экран исполняемой строки с автоматической подстановкой значений переменных
\begin{lstlisting}[language=sh]
set -x
echo $HOME
\end{lstlisting}

		\item {\tt set -n} -- проверка синтаксиса скрипта
\begin{lstlisting}[language=sh]
bash -n test.sh
\end{lstlisting} 


		\item {\tt set -o} -- отключение генерации имен файлов с использованием метасимволов (globbing)
\begin{lstlisting}[language=sh]
echo ~/.*
set -o
echo ~/.*
\end{lstlisting} 
		\framebreak
		\item {\tt set -e} -- остановка выполнения скрипта, если какая-либо команда 
		возвращяет errorstatus не равный 0
			\begin{lstlisting}[language=sh]
#!/bin/bash

set -e
false || true
echo Working
false
echo Still working
			\end{lstlisting} 
	\end{itemize}
\end{frame}

\begin{frame}[fragile]
	\frametitle{Полезные переменные}

	\begin{itemize}
		\item {\tt LINENO} -- номер текущей строки
		\item {\tt FUNCNAME} -- вывод на экран исполняемой строки с автоматической подстановкой значений переменных
		\item {\tt SHLVL} -- уровень вложенности интерпретатора
		\item {\tt PIPESTATUS} -- массив статусов завершения всех команд в pipe
	\end{itemize}
\end{frame}

\begin{frame}[fragile]
	\frametitle{Стек вызовов}

	\begin{itemize}
		\item {\tt caller [N]} -- функция выводит на экран номер строки, имена функции и файла вызывающего скрипта
	\end{itemize}

	\begin{block}{backtrace}
		\begin{lstlisting}[language=sh]
#!/bin/bash
backtrace() {
  echo Backtrace:
  for((i=$SHLVL;i>=0;i--)); do
    caller $i
  done
}
f1() {
  echo Function $FUNCNAME at $LINENO && backtrace
}
f1
		\end{lstlisting}
	\end{block}

\end{frame}

