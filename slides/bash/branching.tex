
\begin{frame}[fragile]
\frametitle{Синтаксис {\bf if}}

	\begin{columns}
		\column{0.4\textwidth}
	
	\begin{lstlisting}[language=bash]
if [ условие1 ]
then
   . . .
elif [ условие2 ]
then
   . . .
else
   . . .
fi
\end{lstlisting}
		\column{0.6\textwidth}
	{\bf Практическое задание:} \\
	\begin{itemize}

		\item с помощью конструкции {\bf if} проверить существует ли файловый объект передаваемый в качестве параметра скрипту
		\item если нет, то создать директорию с таким именем
		\item если cуществует и файл является shell-скриптом, то запустить его
		\item если существует и является директорией, то вывести на экран первых 5 файлов в этой директории
	\end{itemize}
	\end{columns}
\end{frame}

\begin{frame}[fragile]
\frametitle{Условные операторы: case}

	\small
	\begin{columns}
		\column{0.4\textwidth}

		\begin{lstlisting}[language=sh,frame=single]
case "$variable" in 
 pattern1) command1
           command2
          ;;
 pattern2|pattern3)
         command3
         command4
        ;;

esac
		\end{lstlisting}
		\pause

		\column{0.6\textwidth}
		{\normalsize Пример:}

		\begin{lstlisting}[language=sh]
#!/bin/bash

cmd=$1; var=$2
case "$cmd" in 
	--print|--echo)
		echo "Ура, печатаем!" ;;
	abc*|xyz*)
		echo "Странная команда $cmd" ;;
	*)
		echo "Я таких команд не знаю: $cmd" 1>&2
		exit 1 ;;
esac
[ ! -z "$var" ] && echo "переменная \$var=$var"
exit 0
		\end{lstlisting}


	\end{columns}
\end{frame}

\begin{frame}
	\frametitle{Case: практическое задание}

	Написать скрипт, который будет создавать файл, 
	записывать туда текущее время и переданную строку,
	используя при этом следующие аргументы командной строки: 

	\begin{enumerate}
		\item -f или -{}-file <имя файла>\\
			если отсутствует имя файла, то выйти с соответствующим сообщением;
		\item -l или -{}-log <строка>\\
			если отсутствует <строка>, то выйти с соответствующим сообщением;
		\item -a или -{}-append \\
			необязательный флаг, указывающий будет ли файл переписан, либо дописан.
	\end{enumerate}

\end{frame}

\begin{frame}[fragile]
	\frametitle{case + getopts}
	
	Встроенная в bash обработка командной строки.

\begin{lstlisting}[language=sh,frame=single]
while getopts "af:h" Option
do
  case $Option in 
    a) FLAGA=1 ;;
    f) FILE=1
       FILENAME=$OPTARG
       ;;
    h) echo "Usage: $0 [-ah] -f <filename>";;
  esac  
done
shift $((OPTIND-1))
\end{lstlisting}
\end{frame}

\begin{frame}[fragile]
\frametitle{case + getopt}
	
	Внешняя команда для обработки аргументов командной строки (расширенная).

	\small
	\begin{lstlisting}
SHORTOPTS="af:h"
LONGOPTS="flagA, file:, help"
OPTS=$(getopt -o "$SHORTOPTS" -l "$LONGOPTS" -- "$@")
eval set -- "$OPTS"
while [ $# -gt 0 ]
do
    -a|--flagA) FLAGA=1 ;;
    -f|--file) FILE=1
       FILENAME=$OPTARG ;;
    *) echo "Usage: $0 [-a|--flagA] [-h|--help] [-f|--file <filename>]";;
done
	\end{lstlisting}

\end{frame}

\begin{frame}
\frametitle{Практическое задание}
Переписать скрипт из предыдущего задания с использованием {\tt getopt} либо {\tt getopts}.
\end{frame}

\begin{frame}[fragile]
\frametitle{Условные операторы: select}

	\small
	\begin{columns}
		\column{0.4\textwidth}

		\begin{lstlisting}[language=sh,frame=single]
select variable [in list]
do
	command...
	break
done 
		\end{lstlisting}
		\pause
		\column{0.6\textwidth}
		{\normalsize Пример:}

		\begin{lstlisting}
#!/bin/bash
PS3="Какого вы пола? "
select sex in "девочка" "мальчик" "еще не определилось"
do
  echo
  [ "$sex" == "девочка" -o "$sex" == "мальчик" ] && break || echo "Определитесь пожалуйста!"
done
echo "Вы -- $sex"
exit
		\end{lstlisting}

		\pause
		\center{Удалить в select все [in *] и посмотреть на результат работы.}

	\end{columns}
\end{frame}


