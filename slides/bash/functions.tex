\begin{frame}
	\frametitle{Функции}

	\begin{itemize}
		\item Именованные
		\item Неименованные
	\end{itemize}

	Функции в shell могут использоваться как обычные программы, которые:
	\begin{itemize}
		\item Принимают позиционные параметры;
		\item возвращают статус;
		\item Могут использоваться в качестве источника либо приемника 
			при перенаправлениях ввода/вывода.
	\end{itemize}

\end{frame}


\begin{frame}[fragile]
	\frametitle{Функции: синтаксис}
	\begin{itemize}
		\item Классический синаксис: 
			\begin{lstlisting}
function function_name {
command...
} 
			\end{lstlisting}
		\item Портабельный (C-style):
			\begin{lstlisting}
function_name()
{
command...
} 
			\end{lstlisting}

		\item Однострочный:
			\begin{lstlisting}
function_name () { command... ;}
			\end{lstlisting}
  \end{itemize}
\end{frame}

\begin{frame}[fragile]
	\frametitle{Пример (начало)}
	\small
	\begin{lstlisting}
#!/bin/bash

function help {
	echo "Использование: $0 <string>"
	exit 1
}
f1(){
	echo Вызвана функция $0 с $# аргументами
}

f2(){
	while read str; do
		echo $0 прочитана строка: $str
	done
}
	\end{lstlisting}

\end{frame}



\begin{frame}[fragile]
	\frametitle{Пример (окончание)}
	\small
	\begin{lstlisting}
[ $# -eq 0 ] && help

f1 "$@"

{ for ((i=0;i<5;i++));do
	echo $@
done } | f2

exit
	\end{lstlisting}

\end{frame}

\begin{frame}[fragile]
	\frametitle{локальные переменные}

	Переменные объявленные с префиксом {\tt local} видны только внутри объявленного блока.

	\small
	\begin{columns}
		\column{0.5\textwidth}
		\begin{lstlisting}
#!/bin/bash

VAR=test
f1(){
	local VAR=f1
	echo $0: VAR1=$VAR
}
f2(){
    VAR=f2
    echo $0: VAR=$VAR
}
		\end{lstlisting}
		\column{0.5\textwidth}	
		\begin{lstlisting}

echo Before f1: VAR=$VAR
f1
echo Before f2: VAR=$VAR
f2
echo After f2: VAR=$VAR

exit
		\end{lstlisting}
	\end{columns}

\end{frame}

\begin{frame}
	\frametitle{Практическое задание}
	\begin{itemize}
		\item Создать библиотеку функций:
			\begin{itemize}
				\item функция {\tt help};
				\item функция {\tt count\_str} -- подсчитывает количество найденных строк в stdin.
				\item функция {\tt readfile}, в которую передается имя файла в качестве 1 параметра
					и строка, которую необходимо найти, в качестве 2-го.
					Задача функции -- вырезать из файла все пустые строки и комментарии, 
					и с помощью предыдущей функции вернуть количество вхождений искомой строки.
			\end{itemize}
		\item Создать скрипт, который обрабатывает файл переданный через параметр -f или -{}-file;
		\item Включить библиотеку в свой скрипт с помощью {\tt source};
		\item Найти количество сервисов {\tt tcp} в файле {\tt /etc/services};
		\item Найти количество сервисов {\tt udp} в файле {\tt /etc/services};
  \end{itemize}
\end{frame}

