%% Arithmetic

\begin{frame}
  \frametitle{}
  \begin{itemize}
   \item  Конструкция {\tt ((...))}
    \begin{block}{Примеры}
     {\tt (( a=10 )); echo \$(( a++ )); echo \$a; } 
    \end{block}
   \item  {\tt let}
    \begin{block}{Примеры}
     {\tt let a=10; echo \$a; let a+=-2; echo \$a; echo \$(( --a)); echo \$a} 
    \end{block}
   \item  {\tt expr } внешняя команда 
    \begin{block}{Пример}
       { \tt x=\`{}expr \$x + 1\`{} }
    \end{block}
  \end{itemize}
\end{frame}

\begin{frame}[fragile]
\frametitle{Операторы в арифметических выражениях}
\begin{enumerate}
\item Инкременты, декременты {\tt id++, id--, ++id, --id } 
\item Арифметические операторы {\tt **,*,/,\%,+,-} 
\item Побитовые операторы {\verb+ ~,>>,<<,^,&,|+}
\item Операторы сравнения {\tt <=,>=,<,>, ==, !=}
\item Логические операторы {\tt \&\&, || } 
\item Тернарный оператор {\tt expr ? expr : expr }
\item Операторы присваивания
\begin{lstlisting}[language=C]
=, *=, /=, %=,
+=, -=, <<=, >>=,
&=, ^=, |=  
\end{lstlisting}
\end{enumerate} 

  
\end{frame}

\begin{frame}[fragile]
  \frametitle{Упражнение}
 \begin{enumerate}
   \item {\bf Конец света:} Вывести дату конца юниксовых времен ( время $2^{31}-1$) {\tt date -d @<seconds> } 
   \item Проверить результат арифметической операции (( 5>10 ))
 \end{enumerate}
\end{frame}
