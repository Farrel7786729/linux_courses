\begin{frame}{Горячие клавиши}
  \begin{itemize}
    \item \textbf{Tab} -- дополнение текущей команды
      \pause
    \item История команд (Команда {\tt history})
      \begin{itemize}
        \item Клавиши курсора -- навигация по истории
        \item Ctrl-R -- поиск в истории по фрагменту
        \item Ctrl-O (после выполнения вставить следующую команду из истории)
      \end{itemize}
    \item Редактирование командной строки
      \begin{itemize}
        \item Ctrl-W, Ctrl-U -- удалить, начиная от курсора, слово или строку до начала строки
        \item Alt-D, Ctrl-K -- удалить, начиная от курсора, слово или строку до конца строки
      \end{itemize}
    \item Управление терминалом
      \begin{itemize}
        \item Shift-PgUp, Shift-PgDown -- прокрутка терминала
        \item Ctrl-S -- заморозка терминала
        \item Ctrl-Q -- разморозка терминала
        \item Ctrl-L -- очистка терминала
      \end{itemize}
  \end{itemize}
\end{frame}

\begin{frame}{Alias}
  \begin{block}{Bash alias}
    Alias в Bash -- это не более, чем клавиатурное сокращение или своего рода аббревиатура, 
    позволяющая сократить количество нажимаемых клавиш для ввода длинных команд.

    \begin{itemize}
        \item {\tt alias} -- просмотр сокращений
	\item {\tt alias <name>=''cmd1;cmd2''} -- добавление/модификация сокращений \\
	      {\tt alias tls=''netstat -lpnt | grep 127.0.0.1''}
        \item {\tt unalias <cmd>} -- удаление сокращений
    \end{itemize}
  \end{block}

  \pause
  \begin{block}{Практическое задание}
  Создать сокращение для команды {\tt ls} так, чтобы она всегда вызывалась с параметрами {\tt -l} и {\tt -a}.
  \end{block}

\end{frame}
