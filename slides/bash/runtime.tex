

\begin{frame}[fragile]
	\frametitle{Введение}

	\begin{block}{Sha-Bang}
		\begin{lstlisting}
#!/bin/bash
		\end{lstlisting}
	\end{block}

	\begin{block}{Режим совместимости с POSIX}
		\begin{lstlisting}
#!/bin/sh
		\end{lstlisting}

	\end{block}

\end{frame}


\begin{frame}
	\frametitle{Спецсимволы}

	\begin{itemize}
		\item \# -- Вся строка после \# является комментарием
		\item ; -- Разделение команд
		\item : -- NOP оператор (похож на встроенный вызов true)
		\item {\tt source} или {\bf .} -- скрипт выполняется в текущем экземпляре shell
	\end{itemize}

\end{frame}


\begin{frame}
	\frametitle{Спецсимволы}

	\begin{block}{Фигурные скобки и склеивание с помощью ``,``}
		\begin{itemize}
			\item Посмотреть на результат выполнения команды \\
				{\tt echo \{A,B,C\}:\{1,2,3\}}
				\pause
			\item Посмотреть на результат выполнения команды \\
				{\tt ls -l \{,/usr\}/\{bin,sbin\}/*sh}
		\end{itemize}
	\end{block}

\end{frame}


\begin{frame}[fragile]
	\frametitle{Экранирование}

	\begin{columns}
		\column{0.5\textwidth}
		\begin{itemize}
			\item Экранирование одного символа \textbackslash 
			\item Частичное экранирование ''
			\item Полное экранирование '
		\end{itemize}
		\pause
		\column{0.5\textwidth}
		Спецзначения для echo и sed
		\begin{itemize}
			\item \textbackslash{n} -- новая строка
			\item \textbackslash{r} -- возврат каретки
			\item \textbackslash{t} -- табуляция
			\item \textbackslash{v} -- вертикальная табуляция \\
				\small\begin{lstlisting}
echo -e "test \v test \v test"
				\end{lstlisting}
			\item \textbackslash{b} -- перемещение на 1 символ назад
			\item \textbackslash{a} -- звуковой сигнал
			\item \textbackslash{0xxx} -- 8-миричное число
			\item \textbackslash{xXXX} -- 16-ричное число
		\end{itemize}
	\end{columns}

\end{frame}


\begin{frame}
	\frametitle{Подстановка команд}
	
	Синтаксис:

	\begin{itemize}
		\item \`{}command\`{}
		\item \$(command)
	\end{itemize}
	\pause
	\begin{block}{Задание}
		Присвоить переменной LIST результат выполнения команды {\tt ls -1} \\
		Вывести на экран переменную LIST
	\end{block}
\end{frame}

\begin{frame}[fragile]
	\frametitle{Коды возврата}

	Согласно POSIX: 0 -- успех

	Код возврата доступен через переменную \$?

	\pause
	\begin{block}{Пример}
		\begin{lstlisting}
/bin/true; echo $?
/bin/false; echo $?
		\end{lstlisting}
	\end{block}

	Скрипт возвращает код последней команды, поэтому для корректного выхода необходимо использовать {\tt exit}.

\end{frame}


\begin{frame}
	\frametitle{exec}

	Заменяет текущий shell переданной командой. 

	Часто используется для переназначения файловых дескрипторов.

\end{frame}

\begin{frame}
	\frametitle{Управление заданиями}
	
	\begin{itemize}
		\item {\tt bg} -- поместить задачу в фоновое исполнение
		\item {\tt fg} -- ''переключиться'' на задачу
		\item {\tt jobs} -- список активных задач
		\item {\tt wait [pid|id]} -- ждать завершения всех, либо конкретных задач
	\end{itemize}
	\pause
	\begin{block}{Задание}
		\begin{itemize}
			\item Запустить задачу ''{\tt sleep 100 \&}'' в фоновом режиме
			\item Запустить задачу ''{\tt sleep 100}'' на текущей косоли и остановить ее (Ctrl-Z)
			\item С помощью команды ''{\tt jobs}'' посмотреть статус активных задач
			\item С помощью команды ''{\tt bg}'' перевести остановленную задачу в фоновый режим
			\item С помощью команды ''{\tt fg}'' переключиться на одну из фоновых задач
		\end{itemize}
	\end{block}
\end{frame}



\begin{frame}[fragile]
	\frametitle{Группировка}
	
	\begin{block}{Пример}
		\begin{lstlisting}
( echo 1; echo 2) | tee file
		\end{lstlisting}
	\end{block}

	\pause
	\begin{block}{( cmd1; cmd2)}
	    Запускается новый shell
	\end{block}

	\begin{block}{Пример}
		\begin{lstlisting}
TEST=42; (echo $TEST; TEST=0; echo $TEST ); echo $TEST
		\end{lstlisting}
	\end{block}

\end{frame}


\begin{frame}[fragile]
	\frametitle{Перенаправление ввода/вывода}

	\begin{itemize}
		\item ``>'' -- Перенаправление в файл
			\begin{block}{Пример}
				\begin{lstlisting}
echo stdout > test.txt
				\end{lstlisting}
			\end{block}
		\item ``>\&'' -- Перенаправление в другой дескриптор
			\begin{block}{Пример}
				\begin{lstlisting}
(echo stdout; echo stderr >&2) > test.txt
				\end{lstlisting}
			\end{block}
	\end{itemize}

\end{frame}



\begin{frame}[fragile]
	\frametitle{Перенаправление ввода/вывода}

	\begin{itemize}

		\item ``\&>''
			\begin{block}{Пример}
				\begin{lstlisting}
(echo stdout; echo stderr >&2) &> test.txt
				\end{lstlisting}
			\end{block}
		
		\item ``>>'' -- Добавление в файл
			\begin{block}{Пример}
				\begin{lstlisting}
echo stdout >> test.txt
				\end{lstlisting}
			\end{block}
																										                
		\item ``<'' -- Чтение из файла
			\begin{block}{Пример}
				\begin{lstlisting}
cat < test.txt
				\end{lstlisting}
			\end{block}
	\end{itemize}

\end{frame}


\begin{frame}[fragile]
	\frametitle{Перенаправление ввода/вывода}

	\begin{itemize}

		\item ``<<'' -- Here-документ

		\item ``<>'' -- Открывает файловый дескриптор из файла/другого дескритора
			\begin{block}{Пример}
				\begin{lstlisting}
exec 3<>test.txt; echo test >&3;  cat <test.txt
				\end{lstlisting}
			\end{block}
			
		\item ``n<\&-'' -- Закрывает файловый дескриптор
			\begin{block}{Пример}
				\begin{lstlisting}
exec 3<&-; echo test >&3
				\end{lstlisting}
			\end{block}
			
		\item ``|'' -- pipe
	\end{itemize}

\end{frame}


