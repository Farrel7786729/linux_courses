\begin{frame}{Хранение информации о пользователях в системе}

	\begin{block}{\tt /etc/group}
		{\tt  group\_name:password:GID:user\_list}
	\end{block}
	
	\pause

	\begin{block}{\tt /etc/passwd}
		{\tt account:password:UID:GID:GECOS:directory:shell}

		\begin{itemize}
			\item {\tt *} -- пароль не задан
			\item {\tt x} -- пароль задан в файле {\tt /etc/shadows}
		\end{itemize}
	\end{block}

	\pause

	\begin{block}{\tt /etc/shadow}
		\begin{enumerate}
			\begin{columns}
			\column{0.3\textwidth}

			\item login name
			\item encrypted password
			\item date of last password change

			\column{0.3\textwidth}		
			\item minimum password age 
			\item maximum password age
			\item password warning period

			\column{0.3\textwidth}
			\item password inactivity period
			\item account expiration date
			\item reserved field

			\end{columns}
		\end{enumerate}
	\end{block}

\end{frame}

\begin{frame}{Практическое задание}
    \begin{itemize}
		\item Посмотреть права доступа к файлам {\tt group}, {\tt passwd}, {\tt shadow}\\
			{\tt ls -l /etc/{group, passwd, shadow}}
		\item Добавить пользователя и группу и посмотреть изменения в перечисленных файлах
		\item Cоздать пользователя без использования системных утилит (пароль взять такой же, как у существующего пользователя).
    \end{itemize}
	\pause
	 \begin{itemize}
		\item Изменить пароль пользователю с помощью утилиты {\tt passwd}%\\
%			Hint: {\tt /etc/passwdqc.conf}
		\item Сбросить пароль пользователю\\
			Hint: {\tt usermod}
    \end{itemize}
\end{frame}


