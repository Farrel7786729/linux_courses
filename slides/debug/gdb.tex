\begin{frame}
  \frametitle{Базовые команды gdb}
  \begin{columns}
    \begin{column}{0.3\textwidth}
      \begin{itemize}
        \item help
        \item b (break)
        \item c (continue)
        \item s (step)
      \end{itemize}
    \end{column}
    
    \begin{column}{0.3\textwidth}
      \begin{itemize}
        \item p (print)
        \item l (list)
        \item (pt) ptype
        \item watch
      \end{itemize}
    \end{column}

    \begin{column}{0.3\textwidth}
      \begin{itemize}
        \item bt (backtrace)
        \item finish
        \item return
        \item frame
      \end{itemize}
    \end{column}
  \end{columns}
\end{frame}

\begin{frame}
 \frametitle{Обращение с точками останова}
 \begin{columns}
 \column{0.5\textwidth}
 \begin{itemize}
   \item{Варианты установки}
   \begin{itemize}
     \item \texttt{break}
     \item \texttt{break <line>}
     \item \texttt{break <file>:<line>}
     \item \texttt{break <function>}
     \item \texttt{break <address>}
     \item \texttt{break <somewhere> if <condition>}
   \end{itemize}
\end{itemize}
\column{0.5\textwidth}
  \begin{itemize}   
    \item Удаление
     \begin{itemize}
       \item \texttt{clear} 
       \item \texttt{delete}
     \end{itemize}
     \item Временное отключение
     \begin{itemize}
       \item \texttt{disable}
       \item \texttt{enable}
       \item \texttt{ignore <n>}
     \end{itemize}
  \end{itemize}
\end{columns}
\end{frame}

\begin{frame}
 \frametitle{Отладка внешнего процесса}
 \begin{itemize}
   \item \texttt{gdb <file> -p pid}
   \item \texttt{attach} внутри gdb 
 \end{itemize} 
\end{frame}

\begin{frame}
  \frametitle{Упражнение}
  \begin{enumerate}
    \item Собрать программку counter и запустить ее с начальным значением 100
    \item Найти pid запущенной программки и подключиться к ней с gdb
    \item Установить breakpoint внутри цикла
    \item Продолжить выполнение
    \item Удалить breakpoint внутри цикла
    \item Установить breakpoint при i==20
    \item После остановки на breakpoint'е изменить значение howmany из gdb на 30
    \item Продолжить выполнение
  \end{enumerate}
\end{frame}

\begin{frame}
  \frametitle{Работа со стеком и core dump.}
  \begin{center}
    Упражнение
  \end{center}
  \begin{enumerate}
   \item Скомпилировать \texttt{segfault.c}
   \item Установить \texttt{ulimit -c 1000}\\
	   {\small Для некоторых дистрибутивов:\\ {\tt sysctl -w kernel.core\_pattern=/tmp/core-\%e-\%p}}
   \item Запустить segfault, дождаться segmentation fault
   \item Запустить \texttt{gdb ./segfault /tmp/core-....}
   \item Изучить значения переменных в текущем стековом фрейме
   \item Изучить значения переменных в предыдущем стековом фрейме
   \item Исправить код
  \end{enumerate}
%+core dump
\end{frame}

\begin{frame}[fragile]
  \frametitle{Отладочная информация в отдельных файлах (dbg)}
  \begin{center}
    Последовательность создания отдельного файла с отладочными символами
  \end{center}
  \begin{itemize}
    \item \verb+ objcopy --only-keep-debug <exec file> <dbg file>+
    \item \verb+ strip -g <exec file>+
    \item \verb+ objcopy  --add-gnu-debuglink=<dbg file> <exec file> + 
    \item \texttt{gdb: set debug-file-directory}
  \end{itemize} 
\pause
  \begin{center}
   Упражнение
  \end{center}
  \begin{itemize}
    \item Добавить к Makefile дополнительную цель counter.dbg, которая создаст \texttt{counter} (stripped) и \texttt{counter.dbg}
    \item Запустить gdb на файле counter и убедиться, что все работает
  \end{itemize}
\end{frame}  

%\begin{frame}
% \frametitle{gdb server}
%\end{frame}


